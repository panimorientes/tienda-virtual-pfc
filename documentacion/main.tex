%%%%%%%%%%%%%%%%%%%%%%%%%%%%%%%%%%%%%%%%%%%%%%%%%%%%%%%%%%%%%%%%%%%%%%
%% LaTeX PRO TEMPLATE  %%%%%%%%%%%%%%%%%%%%%%%%%%%%%%%%%%%%%%%%%%%%%%%
%%%%%%%%%%%%%%%%%%%%%%%%%%%%%%%%%%%%%%%%%%%%%%%%%%%%%%%%%%%%%%%%%%%%%%

\documentclass[a4paper, 10pt, twoside]{book}

% Codificación UTF8
\usepackage[utf8]{inputenc}
\usepackage[spanish]{babel}
\usepackage[T1]{fontenc}
\usepackage{eurosym}
% Espaciado entre párrafos
\setlength{\parskip}{5pt} 
% Espaciado entre lineas
\usepackage{setspace}
% Hipervínculos dentro del documento
\usepackage{hyperref}
% Reemplazo de las macros para las secciones
\usepackage{titlesec}
% Ubicaciones absolutas
\usepackage{float}
% Encabezado y pie de página
\usepackage{fancybox}
\usepackage{fancyvrb}
% Elementos centrados en página en plantillas book
\usepackage{changepage}
% Tipografía del documento
\usepackage{amsmath}
\usepackage{bookman}  % bookman | newcent | palatino | times
% Para la inserción de imágenes
\usepackage[pdftex]{graphics}
% Configuración de página
\usepackage{a4wide}
\usepackage[a4paper,left=7cm,right=2cm, top=2.75cm, bottom=2.5cm, marginparwidth=2.5cm, marginparsep=1.0cm]{geometry}
\usepackage{enumerate}
% Para colocar bloques de texto en coordenadas absolutas
\usepackage[absolute]{textpos}
% Para poner imágenes PDF de fondo
\usepackage{eso-pic}
% Para utilizar interlineados personalizados
\usepackage{setspace}
\usepackage{pdfpages}
% Macros personalizadas
\usepackage[utf8]{inputenc}
\usepackage{ifthen}
\usepackage{colortbl}
\usepackage{fancyvrb}
\usepackage{captdef}
\usepackage{fancybox}
\usepackage{lettrine}
\usepackage{shadow}
\usepackage{mparhack}
\usepackage{tabularx}
\usepackage{booktabs}

\definecolor{gris}{rgb}{0.3,0.3,0.3}
\definecolor{grisclaro}{rgb}{0.85,0.85,0.85}
\definecolor{grisoscuro}{rgb}{0.2,0.2,0.2}
\definecolor{comentarios}{rgb}{0.15, 0.15, 0.4}
\definecolor{grisconsola}{rgb}{0.85,0.85,0.85}

\renewcommand{\LettrineFontHook}{\color[gray]{0.4}}

\graphicspath{{images/}}
\reversemarginpar

\usepackage{fancyhdr}

%\setlenght\headheight{17pt}
\fancyhf{}
\fancyheadoffset[RO,LE]{0pt}
\fancyheadoffset[LO]{145pt}
\fancyheadoffset[RE]{145pt}
\fancyhead[RO,LE]{[\textbf{\thepage}]}
\fancyhead[LO]{\nouppercase{\rightmark}} % Seccion
\fancyhead[RE]{\leftmark}  % CAPITULO
\renewcommand\headrule{\hrule height 1.5pt width\headwidth \vspace{1mm}}

\usepackage{listings}
\usepackage{courier}
\renewcommand{\lstlistingname}{Listado}

% Para el dibujado de grafos (requiere tener instalado el paquete dot2tex en el sistema)
\usepackage[outputdir={./dot/},autosize]{dot2texi}
\usepackage{tikz}
\usetikzlibrary{shapes,arrows,automata}

% \lstset{
%          basicstyle=\footnotesize\ttfamily, 
%          numbers=left,               
%          numberstyle=\tiny,          
%          numbersep=10pt,              
%          tabsize=2,                  
%          extendedchars=true,         
%          breaklines=true,            
%          keywordstyle=\color{red},
%          frame=b,
%          framerule=1pt,
%          keywordstyle=\color{black}\bfseries,
%          stringstyle=\color{gris}\ttfamily, 
%          showspaces=false,           
%          showtabs=false,             
%          xleftmargin=17pt,
%          framexleftmargin=17pt,
%          framexrightmargin=5pt,
%          framexbottommargin=4pt,
%          commentstyle=\color{comentarios},
%          backgroundcolor=\color{grisclaro},
%          showstringspaces=false, 
%          escapeinside={(*@}{@*)}
%  }

\lstset{tabsize=4,
        showspaces=false,
        showtabs=false,
        frame=b,
        framerule=1pt,
        aboveskip=0.5cm,
        framextopmargin=3pt,
        framexbottommargin=3pt,
        framexleftmargin=18pt,
        framesep=.4pt,
        rulesep=.4pt,
        rulesepcolor=\color{grisoscuro},
        stringstyle=\ttfamily,
        showstringspaces = false,
        basicstyle=\footnotesize\ttfamily,
        commentstyle=\color{gris},
        keywordstyle=\bfseries,
        numbers=left,
        numbersep=6pt,
        numberstyle=\color[cmyk]{0.43, 0.35, 0.35,0.01}\bfseries\scriptsize\ttfamily,
        numberfirstline = true,        
        breaklines=true,
        stepnumber=1,
        backgroundcolor=\color{white},
        xleftmargin=18pt,
        framexrightmargin=0pt,
        xrightmargin=0pt
}

\usepackage{caption}
\DeclareCaptionFont{white}{\fontsize{9}{9}\selectfont\color{white}}
\DeclareCaptionFormat{listing}{\colorbox[cmyk]{0.43, 0.35, 0.35,0.01}{\parbox{0.982\textwidth}{\hspace{15pt}#1#2#3}}}
\captionsetup[lstlisting]{format=listing,labelfont=white,textfont=white, singlelinecheck=false, margin=0pt, font={bf,footnotesize}}

%%%%%%%%%%%%%%%%%%%%%%%%%%%%%%%%%%%%%%%%%%%%%%%%%%%%%%%%%%%%%%%%%%%%%%%%%%%%%%%%

\newcommand{\bigrule}{\vspace{-0.5cm}\hspace{-4.95cm}\titlerule[1.0mm]}

%%%%%%%%%%%%%%%%%%%%%%%%%%%%%%%%%%%%%%%%%%%%%%%%%%%%%%%%%%%%%%%%%%%%%%%%%%%%%%%%

\lstnewenvironment{term}{
  \interlineado{0.92}
  \lstset{
    basicstyle=\footnotesize\bf\ttfamily,
    numbers=none,
    xleftmargin=3.5pt,
    frame=none, 
    breaklines=true, 
    framexleftmargin=3pt,
    framexrightmargin=0pt,
    backgroundcolor=\color{grisclaro},
  }
}{\interlineado{1.0}}

%%%%%%%%%%%%%%% %%%%%%%%%%%%%%%%%%%%%%%%%%%%%%%%%%%%%%%%%%%%%%%%%%%%%%%%%%%%%%%%%


%%%%%%%%%%%%%%%%%%%%%%%%%%%%%%%%%%%%%%%%%%%%%%%%%%%%%%%%%%%%%%%%%%%%%%%%%%%%%%%%

 \lstloadlanguages{% Check Dokumentation for further languages ...
         C, Python
 }

\lstdefinestyle{C}
{language=C, }

\lstdefinestyle{P}
{language=Python, }

\lstdefinestyle{consola}
{
  basicstyle=\footnotesize\bf\ttfamily,
  backgroundcolor=\color{grisconsola},
}

\def\codigofuente#1#2#3#4{
  \interlineado{#4}
  \lstinputlisting[label=#3,caption=#2, style=C]{#1}
  \interlineado{1.0}
}

\lstnewenvironment{terminal}
    {\interlineado{0.92}\lstset{style=consola, numbers=none, frame=none, breaklines=true, frame=l, framexleftmargin=7pt}}
    {\interlineado{1.0}}



\def\comando#1{\texttt{\small{#1}}}

\def\linea#1{\fboxsep=0.8mm\fontsize{7}{7}\selectfont\ovalbox{\texttt{\textbf{#1}}}\normalsize}

\def\botonazo#1{\fboxsep=0.8mm\fontsize{6}{6}\selectfont\ovalbox{\textbf{#1}}\normalsize}


\def\raton#1{
\negthinspace \negthinspace
\begin{minipage}[c][1ex][c]{1em}
   \resizebox{!}{1em}{\includegraphics{botonazos/#1}}
\end{minipage}
}


\def\interfaz#1{
\negthinspace \negthinspace
\begin{minipage}[c][1ex][c]{1em}
   \resizebox{!}{1em}{\linethickness{0.3mm} \frame{\includegraphics{botonazos/#1}}}
\end{minipage}
}















 



%%%%%%%%%%%%%%%%%%%%%%%%%%%%%%%%%%%%%%%%%%%%%%%%%%%%%%%%%%%%%%%%%%%%%%



%% Insercion de teclazos %%%%%%%%%%%%%%%%%%%%%%%%%%%%%%%%%%%%%%%%%%%%%%%%%%%
%% Ejemplo: \teclazo{Ctrl} %%%%%%%%%%%%%%%%%%%%%%%%%%%%%%%%%%%%%%%%%%%%%%%%%

\def\teclazo#1{
    \negthinspace 
    \fontsize{9}{6}\selectfont \ovalbox{\texttt{\textbf{#1}}} \normalsize
    \negthinspace 
}

%% Imagen en el margen %%%%%%%%%%%%%%%%%%%%%%%%%%%%%%%%%%%%%%%%%%%%%%%%%%%%%
%% Ejemplo: \imagenmargen{path}{label}{caption} %%%%%%%%%%%%%%%%%%%%%%%%%%%%

\def\imagenmargen#1#2#3{
    \marginparsep=1cm
    \captionsetup{margin=0pt,font=footnotesize,labelfont=bf}
    \marginparwidth=4cm
    \marginpar{ 
        \resizebox{\marginparwidth}{!}{\includegraphics{#1}}
    }
    \marginpar{
        \vspace{0.2cm}
        \figcaption{#3}\label{#2}
    }
}

%% Nota en el margen %%%%%%%%%%%%%%%%%%%%%%%%%%%%%%%%%%%%%%%%%%%%%%%%%
%% Ejemplo: \notamargen{Texto} %%%%%%%%%%%%%%%%%%%%%%%%%%%%%%%%%%%%%%%

\def\notamargen#1#2{
    \interlineado{1.3}
    \ifodd\thepage
        \marginparsep=1.0cm
    \else
        \marginparsep=1.0cm
    \fi 
    \marginparwidth=4.0cm  
    \marginpar{ 
        \fontsize{9}{8}\selectfont
        \fboxsep=1.5mm
        \vspace{0.2cm} \hspace{-0.08cm} \fbox{#1} \vspace{0.2cm} \\
        \fontsize{8}{7.5}
        \selectfont #2
    }
    \marginparwidth=2.5cm
    \normalsize
    \interlineado{1.0}
}

%% Bloque llamativo %%%%%%%%%%%%%%%%%%%%%%%%%%%%%%%%%%%%%%%%%%%%%%%%%%
%% Uso: \importante{tipo}{Texto} % tipo = warning | info | question

\def\importante#1#2{
    \vspace{3mm}
    \small    
    \fboxrule=0.5pt
    \fboxsep=5mm
    \par\noindent\fcolorbox{grisoscuro}{grisclaro}{\parbox{1.3cm}{\resizebox{1cm}{!}{\includegraphics{./img/iconos/#1.png}}}\parbox{0.81\hsize}{#2}}
    \vspace{3mm}
    \fboxsep=1.5mm
    \normalsize
}

%% Insercion de imagen como figura %%%%%%%%%%%%%%%%%%%%%%%%%%%%%%%%%%%
%% Uso: \imagenhere{nombreFichero}{Ancho}{Descripcion}{Identificador}

\def\imagenhere#1#2#3#4{
  \begin{figure}[h]
    \begin{center}
%     \resizebox{#2\textwidth}{!}{\includegraphics{#1}}
      \resizebox{#2}{!}{\includegraphics{#1}}
      \ifthenelse{\equal{#3}{}}{}{\caption {#3}}
      \label{#4}
    \end{center}
  \end{figure}
}

%% Letra capital %%%%%%%%%%%%%%%%%%%%%%%%%%%%%%%%%%%%%%%%%%%%%%%%%%%%%
%% Uso: \capital{letra}

\def\capital#1{
    \lettrine[lines=3, lhang=0.00, loversize=0.04]{#1}{}  
}

%% Insercion de citas %%%%%%%%%%%%%%%%%%%%%%%%%%%%%%%%%%%%%%%%%%%%%%%%
%% Uso: \cita{frase}{autor}

\def\cita#1#2{
    \vspace{0.2cm}
    \begin{quote}
        #1
        \begin{flushright}
            {\it #2}
        \end{flushright}
    \end{quote}
    \normalsize
}

%% Inserción de código fuente %%%%%%%%%%%%%%%%%%%%%%%%%%%%%%%%%%%%%%%%
%% Uso: \code{lenguaje}{numero_primera_linea}{caption}{archivo_fuente}

\def\code#1#2#3#4{
  \lstinputlisting[language=#1, firstnumber=#2, caption=#3]{#4}
}

%% Cambiar el interlineado %%%%%%%%%%%%%%%%%%%%%%%%%%%%%%%%%%%%%%%%%%%
%% Uso: \interlineado{factor} % factor 1.0 es el normal, 2.0 doble...

\newcommand{\interlineado}[1]{
    \renewcommand{\baselinestretch}{#1}  % -- Cambiamos interlineado
	 \large\normalsize % ---------------------- Para que cambie de verdad
}

%% Imagen de ancho total %%%%%%%%%%%%%%%%%%%%%%%%%%%%%%%%%%%%%%%%%%%%%
%% Uso: \imagenanchototal{path}{caption}{label}

\def\imagenanchototal#1#2#3{
  \begin{figure}[h]
    \ifodd\thepage
      \hspace{-5.2cm}
    \else
      \hspace{-0.1cm}
    \fi 
    \begin{minipage}{17.1cm}
      \centering
      \includegraphics[width=\textwidth]{#1}
      \ifthenelse{\equal{#2}{}}{}{\caption{#2}}
      \ifthenelse{\equal{#3}{}}{}{\label{#3}}
    \end{minipage}
  \end{figure}
}

\def\imagenanchototalpar#1#2#3{
  \begin{figure}[h]
    \hspace{-5.2cm}
    \begin{minipage}{17.1cm}
      \centering
      \includegraphics[width=\textwidth]{#1}
      \ifthenelse{\equal{#2}{}}{}{\caption{#2}}
      \ifthenelse{\equal{#3}{}}{}{\label{#3}}
    \end{minipage}
  \end{figure}
}

\def\imagenanchototalimpar#1#2#3{
  \begin{figure}[h]
    \hspace{-0.1cm}
    \begin{minipage}{17.1cm}
      \centering
      \includegraphics[width=\textwidth]{#1}
      \ifthenelse{\equal{#2}{}}{}{\caption{#2}}
      \ifthenelse{\equal{#3}{}}{}{\label{#3}}
    \end{minipage}
  \end{figure}
}


%%%%



%% Grafos DOT empotrados %%%%%%%%%%%%%%%%%%%%%%%%%%%%%%%%%%%%%%%%%%%%%
%% Uso: FALTA POR HACER UN ENTORNO PARA ESTO. A CONTINUACION PUEDES VER
%% UN EJEMPLO DE USO:

%% \begin{figure}[h]
%% \begin{center}
%%   \begin{dot2tex}[dot,scale=1,mathmode]
%%     digraph G {
%%       rankdir=TB;
%%       d2ttikzedgelabels = false;
%%       node [style="state"];
%%       edge [lblstyle="auto"];
%%       X [label = "\pi"];
%%       A -> B [label = "\displaystyle\frac{x_2}{\sum y_i}"];
%%       A -> D [label = "7"];
%%       A -> X;
%%       X -> D;
%%     }
%%   \end{dot2tex}
%% \caption{Grafo empotrado}
%% \label{grafo01}
%% \end{center}
%% \end{figure}





%% Más cosillas




%% % ------------------ Macro para aniadir un cuadro sombreado ---------------------
%% %       Uso: \begin{sombreado}{Texto...}\end{sombreado} 
%% % ------------------------------------------------------------------------------

%% \newenvironment{sombreado}[1]%
%% {%\vspace{2mm}
%%  \par\noindent
%%  \fboxrule=1pt
%%  \fboxsep=2mm
%%  \addtolength{\hsize}{-2\fboxsep} 
%%  \addtolength{\hsize}{-2\fboxrule}
%%  \fcolorbox{black}{gris}{\parbox{\hsize}{#1}}
%%  %\vspace{2mm}
%%  }{}



%% % ------------------ Macro para insertar una subimagen con marco -------------------
%% %       Uso: \subimagenmarco{nombreFichero}{Ancho}{Etiqueta}{Identificador}
%% % -----------------------------------------------------------------------------
%% \def\subimagenmarco#1#2#3#4{
%%  \subfigure[#3]{
%%    \resizebox{#2\textwidth}{!}{\linethickness{0.3mm} \frame{\includegraphics{#1}}}
%%    \label{'#4'}
%%  }	
%% }


%% \newenvironment{intro}[1]%
%% {\vspace{-15mm}
%%  \small
%%  \par\noindent
%%  \fboxrule=1pt
%%  \fboxsep=4mm
%%  \fcolorbox{white}{white}{\parbox{0.92\hsize}{\textcolor{grisoscuro}{$\triangle$ \textit{#1}}}}
%%  \vspace{5mm}
%%  \fboxsep=1.5mm
%%  \normalsize
%%  }{}

%% % ------------------ Macro para insertar una imagen cedida por alguien --------
%% %       Uso: \imagen{nombreFichero}{Ancho}{Etiqueta}{Identificador}
%% % -----------------------------------------------------------------------------
%% \def\imagenCopyright#1#2#3#4#5{
%%  \begin{figure}[here]
%%  \begin{center}
%%    \resizebox{#2\textwidth}{!}{\includegraphics{#1}}
%%  \fontsize{7}{7}\selectfont
%%  \begin{flushright}
%% 	 \vspace{-1mm}
%% 	 \textsf{#5}
%%  \end{flushright}
%%  \normalsize 
%%  \vspace{-3mm}
%%  \caption {#3}
%%  \label{'#4'}
%%  \end{center}
%%  \end{figure}
%% }


%% % ------------------ Macro para insertar una imagen con marco
%% %       Uso: \imagenmarco{nombreFichero}{Ancho}{Etiqueta}{Identificador}
%% % -----------------------------------------------------------------------------
%% \def\imagenmarco#1#2#3#4{
%%  \begin{figure}[here]
%%  \begin{center}
%%    \resizebox{#2\textwidth}{!}{\linethickness{0.3mm} \frame{\includegraphics{#1}}}
%%  \ifthenelse{\equal{#3}{}}{}{\caption {#3}}
%%  \label{'#4'}
%%  \end{center}
%%  \end{figure}
%% }

%% \def\imagenanchototal#1#2#3#4#5{
%%   \captionsetup{margin=4pt,font=small,labelfont=bf}
%%  \begin{figure}
%%  \ifthenelse{\equal{#5}{par}}{\hspace{0cm}}{\hspace{-5.3cm}} 
%%  \resizebox{#2}{!}{\includegraphics{#1}}
%%  \ifthenelse{\equal{#3}{}}{}{\caption {#3}}
%%  \label{#4}
%%  \end{figure}
%% }

%% %% ------------------ Macro para insertar una imagen ---------------------------
%% %       Uso: \imagen{nombreFichero}{Ancho}{Etiqueta}{Identificador}
%% % -----------------------------------------------------------------------------
%% \def\imagen#1#2#3#4{
%%   \captionsetup{margin=4pt,font=small,labelfont=bf}
%%  \begin{figure}
%%  \begin{center}
%%    \resizebox{#2\textwidth}{!}{\includegraphics{#1}}
%%  \ifthenelse{\equal{#3}{}}{}{\caption {#3}}
%%  \label{#4}
%%  \end{center}
%%  \end{figure}
%% }

%% % ------------------ Macro para aniadir codigo fuente --------------------
%% %       Uso: \begin{VerbatimEnvironment}[label=\normalsize\textrm{\textbf{titulo}}]
%% % -----------------------------------------------------------------------

%% \DefineVerbatimEnvironment
%%  {MiVerbatim}{Verbatim}
%%  {frame=single, framerule=.3mm, framesep=4mm, samepage=false,
%%    rulecolor=\color{gris}, baselinestretch=.7, fontsize=\small, 
%%    numbers=left, numbersep=2mm, xleftmargin=2mm, xrightmargin=2mm}
 
%% %%%%%%%%%%%%%%%%%%%%% IMAGEN EN EL MARGEN %%%%%%%%%%%%%%%%%%%%%%%%%%%%%%%

%% \def\slide#1{
%%   \ifodd\thepage \marginparsep=2.5cm \else \marginparsep=1cm \fi 
%%   \marginpar{ 
%%     \resizebox{1.6\marginparwidth}{!}{\linethickness{0.3mm} \frame{\includegraphics{#1}}}
%%   }
%% }


% Personalización de los títulos de los capítulos
\titleformat{\chapter}[display]{\bfseries\Huge}{
  \filleft
  \vspace{0.5cm}
  \Large\chaptertitlename\ % "Capítulo" o "Apéndice"
  \textcolor{gris}{\fontsize{90}{90}\selectfont\thechapter}\normalsize} % número de capítulo
{0cm} % espacio mínimo entre etiqueta y cuerpo
{\filleft} % texto del cuerpo alineado a la derecha
[\vspace{0.5mm} \bigrule] % después del cuerpo, dejar espacio vertical y trazar línea horizontal gruesa

% Para crear índices de términos
\usepackage{makeidx}
\makeindex

%%%%%%%%%%%%%%%%%%%%%%%%%%%%%%%%%%%%%%%%%%%%%%%%%%%%%%%%%%%%%%%%%%%%%%
%% INFORMACION DEL DOCUMENTO %%%%%%%%%%%%%%%%%%%%%%%%%%%%%%%%%%%%%%%%%
 
\def\titulo{SISTEMA DE TIENDA ONLINE}
\def\subtitulo{Ingeniería del Software\\2011-2012}
\def\autor{Sergio G., Gabriel A., David P., Jorge M.}
\def\descripcionautor{Ingeniería Superior de Informática}
\def\email{perico@dominio.es}
\def\web{www.sgmonda.com}
\def\telefono{661783482}
\def\ano{2011}

%%%%%%%%%%%%%%%%%%%%%%%%%%%%%%%%%%%%%%%%%%%%%%%%%%%%%%%%%%%%%%%%%%%%%%
%%%%%%%%%%%%%%%%%%%%%%%%%%%%%%%%%%%%%%%%%%%%%%%%%%%%%%%%%%%%%%%%%%%%%%

\title{\titulo}
\author{\autor}

\begin{document}

%%%%%%%%%%%%%%%%%%%%%%%%%%%%%%%%%%%%%%%%%%%%%%%%%%%%%%%%%%%%%%%%%%%%%%
%% PORTADA %%%%%%%%%%%%%%%%%%%%%%%%%%%%%%%%%%%%%%%%%%%%%%%%%%%%%%%%%%%

%%%%%%%%%%%%%%%%%%%%%%%%%%%%%%%%%%%%%%%%%%%%%%%%%%%%%%%%%%%%%%%%%%%%%%
% Plantilla LaTeX para Ingeniería 3.1
% Sergio García Mondaray
% www.yakiboo.net
%%%%%%%%%%%%%%%%%%%%%%%%%%%%%%%%%%%%%%%%%%%%%%%%%%%%%%%%%%%%%%%%%%%%%%


%%%%%%%%%%%%%%%%%%%%%%%%%%%%%%%%%%%%%%%%%%%%%%%%%%%%%%%%%%%%%%%%%%%%%%%%%%%%%%
% PORTADA
%%%%%%%%%%%%%%%%%%%%%%%%%%%%%%%%%%%%%%%%%%%%%%%%%%%%%%%%%%%%%%%%%%%%%%%%%%%%%% 

\definecolor{black}{RGB}{0,0,0}
\definecolor{dark}{RGB}{50,50,50}
\definecolor{gray}{RGB}{100,100,100}
\definecolor{lightgray}{RGB}{200,200,200}
\definecolor{white}{RGB}{255,255,255}

\thispagestyle{empty}

\AddToShipoutPicture*{
  \put(0,0){\includegraphics{./cfg/front-A/front-A.png}}
}

\textbf{}\\[5cm]
\begin{adjustwidth*}{-140pt}{} % Margenes izquierdo y derecho

  \begin{center}

    \begin{spacing}{1}
      \begin{Huge}\textbf{\textcolor{white}{\titulo}}\end{Huge}
    \end{spacing}
    \begin{Large}{\it \textbf{\textcolor{white}{\subtitulo}}}\end{Large}\\[0.2cm]
    \textcolor{white}{*\ *\ *}\\[0.2cm]
    \begin{large}\textcolor{white}{\bf Sergio García Mondaray\\Gabriel Alises Cano\\David Perez Zaba\\Jorge Merino Garcia}\end{large}\\

    \textbf{}\\[7cm]

  \end{center}
\end{adjustwidth*}


\begin{textblock}{16}(0, 11)
  \begin{center}
    \includegraphics[scale=0.5]{./img/uclm01.jpg}

    {\bf Universidad de Castilla-La Mancha}\par
    Escuela Superior de Informática\\de Ciudad Real
  \end{center}

\end{textblock}








\pagestyle{empty}

% \textbf{}\\[5cm]
% \begin{adjustwidth*}{-140pt}{} % Margenes izquierdo y derecho

%   \begin{center}

%     \begin{huge}\textbf{\titulo}\end{huge}\\[0.4cm]
%     \begin{Large}{\it \textbf{\subtitulo}}\end{Large}\\[0.9cm]
%     \begin{large}\MakeUppercase{\autor}\end{large}\\[0.5cm]

%       \textbf{}\\[7cm]

%       \includegraphics[scale=0.5]{./img/uclm01.jpg}

%       {\bf Universidad de Castilla-La Mancha}\par
%       Escuela Superior de Informática\\de Ciudad Real
%   \end{center}
% \end{adjustwidth*}

\newpage
\mbox{}
\vspace{16cm}
\begin{adjustwidth*}{-100pt}{50pt} % Margenes izquierdo y derecho
  {\bf\autor}\\[0.1cm]
  \descripcionautor\\[0.3cm]
%  \begin{tabular}{ll}
%    {\it e-mail} & {\tt\email} \\[0.1cm]
%    {\it Teléfono} & {\tt\telefono} \\[0.1cm]
%    {\it Web} & {\tt\web}\\[0.1cm]
%  \end{tabular}

  \vspace{0.5cm}
  \begin{small}
    \copyright~ Los autores del proyecto. Se permite la copia, distribución y/o
modificación de este documento bajo los términos de la licencia de
documentación libre GNU, versión 1.1 o cualquier versión posterior publicada
por la {\em Free Software Foundation}, sin secciones invariantes. Puede
consultar esta licencia en http://www.gnu.org. \\[0.2cm]
Este documento fue compuesto con \LaTeX{}, y ha sido desarrollado a partir de una plantilla en la que han trabajado Carlos González Morcillo y Sergio García Mondaray.

  \end{small}

\end{adjustwidth*}

%%%%%%%%%%%%%%%%%%%%%%%%%%%%%%%%%%%%%%%%%%%%%%%%%%%%%%%%%%%%%%%%%%%%%%

% \chapter*{Agradecimientos}\label{CAPAgradecimientos}
% Lorem ipsum dolor sit amet, consectetur adipiscing elit. Suspendisse sed arcu libero. Nulla vel magna turpis. Praesent hendrerit convallis vehicula. Aliquam lectus arcu, laoreet eget malesuada auctor, pulvinar sed metus. Aliquam in justo sem. Praesent egestas ante quis magna consequat vel condimentum justo mattis. Aenean elementum condimentum viverra. Fusce pellentesque accumsan interdum. Ut dignissim vehicula diam a semper. Suspendisse a rutrum lectus. Donec pharetra tempor lobortis. Sed orci lacus, facilisis et vestibulum at, convallis eget velit. Aliquam convallis placerat lectus non tempus. Maecenas vel lectus vitae eros dapibus egestas. In imperdiet odio at nunc viverra mattis. Quisque aliquam iaculis odio, ut imperdiet risus auctor non. 

Curabitur libero urna, placerat non sodales a, molestie fringilla massa. Aenean malesuada tellus at orci malesuada in tempus risus tincidunt. Vestibulum ante ipsum primis in faucibus orci luctus et ultrices posuere cubilia Curae; Vestibulum ante ipsum primis in faucibus orci luctus et ultrices posuere cubilia Curae; Maecenas pretium facilisis nisl eget sollicitudin. Praesent vitae nisl. 


% \chapter*{Resumen}\label{CAPResumen}
% Lorem ipsum dolor sit amet, consectetur adipiscing elit. Suspendisse sed arcu libero. Nulla vel magna turpis. Praesent hendrerit convallis vehicula. Aliquam lectus arcu, laoreet eget malesuada auctor, pulvinar sed metus. Aliquam in justo sem. Praesent egestas ante quis magna consequat vel condimentum justo mattis. Aenean elementum condimentum viverra. Fusce pellentesque accumsan interdum. Ut dignissim vehicula diam a semper. Suspendisse a rutrum lectus. Donec pharetra tempor lobortis. Sed orci lacus, facilisis et vestibulum at, convallis eget velit. Aliquam convallis placerat lectus non tempus. Maecenas vel lectus vitae eros dapibus egestas. In imperdiet odio at nunc viverra mattis. Quisque aliquam iaculis odio, ut imperdiet risus auctor non.

Sed purus quam, venenatis vitae faucibus nec, condimentum et tellus. Nam a velit orci. Aenean vel turpis est, non congue purus. Praesent pulvinar nulla ut sem blandit porta a id libero. Donec convallis, dui non dignissim auctor, libero elit mattis leo, a imperdiet ipsum erat vitae magna. Pellentesque eu massa vel dolor ultricies feugiat. In hac habitasse platea dictumst. Etiam a mattis mauris. Duis vestibulum eros in lorem condimentum tristique. Suspendisse pharetra justo purus. Nunc accumsan nibh ac tellus mattis et viverra nisi sodales. Donec vitae vehicula libero. Aenean tortor risus, imperdiet id dignissim nec, ornare eget mi. Maecenas vestibulum varius leo a malesuada. Pellentesque rhoncus risus a sapien sagittis pulvinar. Suspendisse leo sem, accumsan placerat elementum sit amet, vulputate ac dolor. Nulla eu lectus tellus. Suspendisse lobortis suscipit molestie. Duis malesuada odio in dui faucibus pharetra sollicitudin odio fringilla.

Curabitur libero urna, placerat non sodales a, molestie fringilla massa. Aenean malesuada tellus at orci malesuada in tempus risus tincidunt. Vestibulum ante ipsum primis in faucibus orci luctus et ultrices posuere cubilia Curae; Vestibulum ante ipsum primis in faucibus orci luctus et ultrices posuere cubilia Curae; Maecenas pretium facilisis nisl eget sollicitudin. Praesent vitae nisl. 

%\pagenumbering{arabic}

\tableofcontents 
\listoffigures
%\listoftables

\pagestyle{fancy}

\chapter{Introducción}\label{CAPIntroduccion}
\section{Plan de proyecto}

Para nuestro proyecto tendremos reuniones de seguimiento al final de cada semana para ver que se han realizado las tareas asignadas a cada uno. Cada dos semanas quedaremos para debatir dudas sobre el proyecto y asignar nuevas tareas.

Una vez tengamos una cantidad considerable de información, David Antonio Pérez Zaba revisará el proyecto, y lo corregirá si fuese necesario.

\subsection{Estructura organizativa}

Los roles que se van a seguir para la realización de la práctica, a fin de que todos los integrantes del grupo participen en al menos un flujo de trabajo, son los siguientes:

\begin{enumerate}

\item \textbf{Fase de Inicio:}
  \begin{itemize}
  \item Requisitos:
    \begin{itemize}
    \item Sergio García Mondaray.
    \end{itemize}

  \item Análisis:
    \begin{itemize}
    \item Gabriel Alises Cano.
    \end{itemize}
    
  \item Diseño:
    \begin{itemize}
    \item David Antonio Pérez Zaba.
    \end{itemize}

 \item Implementación:
    \begin{itemize}
    \item Jorge Merino García.
    \end{itemize}
  \end{itemize}

\item \textbf{Fase de Elaboración:}
  \begin{itemize}
  \item Requisitos:
    \begin{itemize}
    \item Gabriel Alises Cano.
    \end{itemize}

  \item Análisis:
    \begin{itemize}
    \item David Antonio Pérez Zaba.
    \end{itemize}
    
  \item Diseño:
    \begin{itemize}
    \item Jorge Merino García.
    \end{itemize}

  \item Implementación:
    \begin{itemize}
    \item Sergio García Mondaray.
    \end{itemize}

  \item Pruebas:
    \begin{itemize}
    \item Gabriel Alises Cano.
    \end{itemize}
  \end{itemize}

\item \textbf{Fase de Construcción:}
  \begin{itemize}
  \item Requisitos:
    \begin{itemize}
    \item David Antonio Pérez Zaba.
    \end{itemize}

  \item Análisis:
    \begin{itemize}
    \item Jorge Merino García.
    \end{itemize}
    
  \item Diseño:
    \begin{itemize}
    \item Sergio García Mondaray.
    \end{itemize}

  \item Implementación:
    \begin{itemize}
    \item Gabriel Alises Cano.
    \end{itemize}

  \item Pruebas:
    \begin{itemize}
    \item David Antonio Pérez Zaba.
    \end{itemize}
  \end{itemize}

\item \textbf{Fase de Transición:}
  \begin{itemize}

  \item Análisis:
    \begin{itemize}
    \item Jorge Merino García.
    \end{itemize}
    
  \item Diseño:
    \begin{itemize}
    \item David Antonio Pérez Zaba
    \end{itemize}

  \item Implementación:
    \begin{itemize}
    \item Gabriel Alises Cano.
    \end{itemize}

  \item Pruebas:
    \begin{itemize}
    \item Jorge Merino García.
    \item Sergio García Mondaray.
    \end{itemize}

  \end{itemize}
\end{enumerate}


\subsection{Responsabilidades}
A nivel de asignatura los profesores se encargarán de proporcionar un sistema de pago externo a nuestra tienda, de manera que se asigna al profesor la responsabilidad de implementación de esta terea.
%\paragraph*{}

Dentro de nuestro grupo al tener repartidos los flujos de trabajo entre cada uno de nuestros integrantes todos tendremos que tener las responsabilidades acordes al estudio de los requisitos, análisis, diseño, implementación y pruebas en distintos momentos del desarrollo de la aplicación. Siendo siempre el jefe de proyecto David Antonio Pérez Zaba.

%\paragraph*{}
El grupo tendrá la responsabilidad de entregar una aplicación funcional el 15 de Noviembre de 2011 a más tardar, con una entrega parcial para el 14 de Octubre de 2011.

\subsection{Gestión de Riesgos}
Los principales riesgos a los que nos enfrentamos a la hora de realizar la aplicación son los siguientes:
\begin{itemize}
%\item No conseguir terminar la aplicación antes de la fecha de entrega.
\item Errores y tiempo de aprendizaje superior al estimado a la hora de utilizar una tecnología que desconocemos como Apache Tomcat.

\begin{itemize}
\item Probabilidad: Alta.

\item Efecto: Tolerable. Si nos retrasamos por no saber configurar estas herramientas, tendremos menos tiempo para elaborar la práctica.

\end{itemize}

\item Problemas y conflictos al utilizar varias personas el control de versiones que hemos elegido (Mercurial), como por ejemplo borrar el progreso de algún compañero o tu propio progreso.

\begin{itemize}
\item Probabilidad: Alta.

\item Efecto: Tolerable, si el sistema tiene el control de versiones actualizadas.

\end{itemize}


\item Dificultades para aprender correctamente HTML,  el lenguaje JavaScript, CSS, y cómo integrar Java con HTML.

\begin{itemize}
\item Probabilidad: Alta.

\item Efecto: Serio. Habrá que dedicar demasiado tiempo al principio para aprender, y puede que no se utilice correctamente debido al poco tiempo que hay para practicar ante de entregar la práctica.

\end{itemize}


\item El tiempo que se ha estimado para desrrollar la práctica se ha subestimado.

\begin{itemize}
\item Probabilidad: Alta.

\item Efecto: Catastrófico, lo que llevará a una sobrecarga de horas de trabajo y posible pérdida de calidad de las entregas.

\end{itemize}


\item Dificultades a la hora de planficar las reuniones, o falta de tiempo de algún miembro por los distintos itinerarios de asginaturas que poseen los miembros del grupo.
\item Errores por no conseguir enfocar el trabajo acorde con la metodología propuesta, es decir, el proceso unificado de desarrollo.

\begin{itemize}
\item Probabilidad: Media.

\item Efecto: Catastrófico. Posibles cambios en el análisis harán que haya que modificar el diseño, y sucesivas modificaciones en la implementación y pruebas.

\end{itemize}


\item Algún miembro del equipo abandona el grupo, o se cambia a otro.

\begin{itemize}
\item Probabilidad: Muy baja.

\item Efecto: Catastrófico. Habría que reorganizar el equipo buscando algún otro integrante o se podrucirá un retraso muy grande en la entrega de la práctica.

\end{itemize}


\item Algún miembro del equipo se encenutra temporalmente indispuesto por alguna enfermedad, o problema personal.

\begin{itemize}
\item Probabilidad: Media.

\item Efecto: Catastrófico. Habrá un sobrecarga de trabajo para los demás miembros del equipo.

\end{itemize}


\item Se da algún error o se han olvidado datos en el análisis y hay que volver a hacer el diseño.

\begin{itemize}
\item Probabilidad: Media.

\item Efecto: Tolerable, siempre y cuando los cambios no sean demasiado grandes, y se pueda solucionar en la siguiente interación.

\end{itemize}


\item Poca coordinación entre los miembros del equipo, por falta de trabajo del jefe de proyecto.

\begin{itemize}
\item Probabilidad: Baja.

\item Efecto: Tolerable. Se habrá creado trabajo repetido, o con poca cohesión.

\end{itemize}


\end{itemize}

\subsection{Métodos,  Herramientas,  Técnicas}
Para el desarrollo de la aplicación utilizaremos el proceso unificado de desarrollo, es decir, seguiremos un proceso iterativo e incremental para corrección de posibles errores en distintas fases.

%\paragraph*{}
Para el control de versiones utilizaremos mercurial para tener de una manera más organizada y poder realizar cambios en los archivos y tener toda la aplicación centralizada en un repositorio que podamos clonar desde cualquier computador con conexión a internet. La técnica que utilizaremos para actualizar el repositorio es ejecutar la instrucción pull antes de comenzar a utilizar cualquier archivo y así trabajar con la última versión de nuestra aplicación. Cuando terminemos nuestra sesión de trabajo notificaremos los cambios con la instrucción ci y posteriormente los subiremos al repositorio con la instrucción push.

%\paragraph*{}
Para implementar la funcionalidad de la aplicación utilizaremos diferentes tecnologías como las que siguen:
\begin{itemize}
\item Para el despliegue de la aplicación utilizaremos Tomcat6.0
\item Para la implementación de la aplicación utilizaremos JavaScript, Java, HTML y, Eclipse como entorno de desarrollo.
\item Para el diseño y análisis de requisitos utilizaremos Visual Paradigm, y Quick Sequence Diagram Editor.
\item Para la planificación de proyecto utilizaremos  Microsoft Project.
\item Para la generación de la documentación utilizaremos Latex.
\item Para la maquetación web utilizaremos plantillas css.
\item Para la coordinación del grupo, utilizaremos el servidor de \url{bitbucket.org} (wiki,rss) y el correo electrónico.
\end{itemize}

\notamargen{Diagrama de Gantt}{El diagrama de Gantt asociado al listado de tareas será mostrado en posteriores entregas.}

\subsection{Lista de tareas}

El conjunto de las tareas a desarrollar y su duración aproximada, así como las dependencias entre las tareas, hitos,... se encuentran en el anexo \ref{tareas}.

% diagrama de Gantt.
% \begin{center}
% \begin{figure}[H]
% \includegraphics[scale = 0.6 ]{./gantt.png}
% \end{figure}
% \end{center}


\chapter{Fase de inicio}\label{CAPInicio}
\section{Enunciado del problema}
\label{enunciado}

Se	  debe	  desarrollar	  una	  tienda	  virtual	  que	  permita	  la	  venta	  de	  artículos	  a	  clientes.	  
Para	  ello,	  la	  tienda	  contará	  con	  una	  base	  de	  datos	  que	  permita	  gestionar	  los	  
productos	  que	  vende,	  sus	  clientes,	  las	  ventas,	  etc.

	  	  
El sistema muestra los productos en stock, destacando productos destacados, recomendados, ofertas, etc. Los	  clientes	  accederán	  a	  la	  tienda	  mediante	  una	  página	  web,	  seleccionarán	  los	  
artículos	  que	  desean	  comprar, seleccionarán la forma de pago de entre los posibles métodos	(tarjeta	  
de	  crédito, tarjeta de  débito  o	  pago externo tipo PayPal) y, por último, confirmarán el pedido. Un pedido completado puede cancelarse en las siguientes 24 horas después del pago.

Una persona puede darse de alta como usuario, identificarse como usuario (o administrador) y añadir o eliminar artículos al carro, ver el carro de la compra, buscar artículos, recuperar la contraseña y ver críticas de los productos. Una persona identificada puede, además, valorar y añadir críticas sobre productos, finalizar una compra y confirmar un pedido (también cancelarlo en las 24 horas siguientes al mismo), seleccionar la forma de pago, modificar sus datos de usuario, darse de baja como cliente, cerrar sesión, consultar el estado e historial de sus pedidos, añadir y borrar artículos de una lista de favoritos, suscribirse y darse de baja de listas de productos. Por último, un administrador puede dar de alta y baja productos, así como modificarlos.

\section{Primer análisis de requisitos}

Estudiemos, llegados a este punto, los requisitos que pueden extraerse del enunciado del problema. Para facilitar la visualización y posterior análisis de los mismos, los clasificaremos en los siguientes grupos:

\begin{enumerate}
\item Requisitos del sistema.
\item Requisitos de visitante.
\item Requisitos de cliente identificado.
\item Requisitos de administrador.
\end{enumerate}

\subsection{Requisitos del sistema}

\begin{itemize}
\item El sistema ofrecerá recomendaciones personalizas a los clientes identificados, en función del historial de sus pedidos.
\item El sistema mostrará los artículos en stock.
\end{itemize}

\subsection{Requisitos de visitante}

Considerando como ``visitante'' al usuario que entra en la web directamente, diferenciamos sus diferentes requisitos:

\begin{itemize}
\item Un visitante puede identificarse para iniciar sesión como un cliente corriente.
\item Un visitante puede identificarse para iniciar sesión como administrador.
\item Un visitante puede darse de alta (registrarse) como cliente, para poder realizar compras.
\item Los visitantes pueden añadir productos al carro.
\item Los visitantes pueden eliminar productos del carro.
\item Un visitante puede mostrar el contenido actual del carro.
\item Los visitantes pueden buscar artículos.
\item Un visitante puede recuperar contraseña, en el caso de que tenga una cuenta y no la recuerde.
\item Los visitantes pueden ver críticas sobre los productos.
\end{itemize}

\subsection{Requisitos de cliente identificado}

En caso de que un visitante inicie sesión, pasa a ser un cliente identificado. Los requisitos asociados a un cliente identificado son los siguientes:

\begin{itemize}
\item Los clientes pueden finalizar una compra, pasando al proceso de pago.
\item Los clientes pueden seleccionar la forma de pago, después de finalizar una compra.
\item Los clientes identificados pueden confirmar un pedido, que será el último paso del proceso de compra (tras seleccionar la forma de pago).
\item Los clientes pueden modificar sus datos de usuario.
\item Un cliente puede darse de baja, produciendo el borrado de su cuenta.
\item Los clientes pueden ver el historial de compras.
\item Los clientes pueden añadir un artículo a favoritos.
\item Los clientes pueden ver el listado de sus artículos favoritos.
\item Los clientes pueden suscribirse al newsletter.
\item Los clientes pueden borrarse del newsletter.
\item Los clientes pueden borrar un artículo de la lista de favoritos.
\item Los clientes pueden cancelar un pedido completado (dentro del margen de tiempo).
\item Los clientes pueden escribir críticas de los productos, siempre que se han comprado, claro.
\item Los clientes identificados pueden cerrar su sesión.
\item Un cliente puede consultar el estado de sus pedidos.
\end{itemize}

\subsection{Requisitos de administrador}

\notamargen{Requisitos comunes}{
Se han clasificado los requisitos en función del grupo al que son más afines. Con el fin de no repetir ningún requisito, cabe destacar que un administrador también podría cerrar sesión, aunque no se haya indicado explícitamente. Los requisitos de administrador listados son los exclusivos de dicho rol.
}

Los requisitos asociados al rol de administrador son los siguientes:

\begin{itemize}
\item Un administrador puede dar de alta un artículo.
\item Un administrador puede dar de baja un artículo.
\item Un administrador puede modificar un artículo.
\end{itemize}

\section{Requisitos funcionales}

Con todo lo anterior, los requisitos funcionales de los que partiremos y en los que basaremos nuestro desarrollo son los siguientes:

\begin{enumerate}[{\bf RF-1}]
\item 
  \begin{itemize}
  \item \textit{Descripción}: El sistema debe ofrecer recomendaciones personalizadas a los clientes identificados, en función del historial de sus pedidos.
  \item \textit{Importancia}: Condicional.
  \item \textit{Validez}:
    \begin{itemize}
    \item \textit{Medible}: Para cada usuario, el sistema deberá encontrar un listado con productos afines.
    \item \textit{Alcanzable}: La web mostrará un listado con los productos recomendados y sus descripciones.
    \item \textit{Relevante}: Este requisito aportaría un componente de dinamismo al sitio web, posibilitando, además, un aumento de ventas.
    \end{itemize}
  \end{itemize}

\notamargen{Sobre los requisitos}{
La lista de requisitos se ha elaborado siguiendo las recomendaciones del estándar IEEE 830-1998.
}

\item 
  \begin{itemize}
  \item \textit{Descripción}: El sistema debe mostrar los artículos en stock.
  \item \textit{Importancia}: Esencial.
  \item \textit{Validez}:
    \begin{itemize}
    \item \textit{Medible}: El sistema deberá ofrecer una lista con todos los productos disponibles en un momento dado.
    \item \textit{Alcanzable}: La web mostrará un listado con los productos existentes en stock, así como sus precios y descripciones.
    \item \textit{Relevante}: Resulta esencial para la tienda, ya que sin mostrar los productos los usuarios no podrían inspeccionarlos y saber si quieren comprarlos.
    \end{itemize}
  \end{itemize}

\item 
  \begin{itemize}
  \item \textit{Descripción}: Un visitante puede identificarse para iniciar sesión como un cliente registrado.
  \item \textit{Importancia}: Esencial.
  \item \textit{Validez}:
    \begin{itemize}
    \item \textit{Medible}: Dado un nombre de usuario y una contraseña, el sistema iniciará la sesión de dicho usuario, si es que existe y la contraseña es correcta.
    \item \textit{Alcanzable}: La web pondrá a disposición del usuario un formulario de autenticación para iniciar sesión.
    \item \textit{Relevante}: Sin este requisito no podríamos saber la identidad del usuario. Por lo tanto, es esencial llevarlo a cabo.
    \end{itemize}
  \end{itemize}


\item 
  \begin{itemize}
  \item \textit{Descripción}: Un visitante puede identificarse para iniciar sesión como administrador.
  \item \textit{Importancia}: Esencial.
  \item \textit{Validez}:
    \begin{itemize}
    \item \textit{Medible}: Dado un nombre de usuario y una contraseña, el sistema iniciará la sesión de dicho usuario (en modo administrador), si es que existe, la contraseña es correcta, y dicho usuario tiene permisos de administración.
    \item \textit{Alcanzable}: La web pondrá a disposición del usuario un formulario de autenticación para iniciar sesión.
    \item \textit{Relevante}: Sin este requisito no podríamos saber la identidad del usuario.
    \end{itemize}
  \end{itemize}


\item 
  \begin{itemize}
  \item \textit{Descripción}: Un visitante puede darse de alta (registrarse) como cliente, para poder realizar compras, mantener listas de favoritos, etc.
  \item \textit{Importancia}: Esencial.
  \item \textit{Validez}:
    \begin{itemize}
    \item \textit{Medible}: La web pondrá a disposición del usuario un formulario de creación de cuenta en la tienda.
    \item \textit{Alcanzable}: El sistema permitirá la creación de un cliente a partir de una información básica (nombre de usuario, email y password).
    \item \textit{Relevante}: Es fundamental ofrecer la posibilidad de que los usuarios se registren sólos.
    \end{itemize}
  \end{itemize}


\item 
  \begin{itemize}
  \item \textit{Descripción}: Los visitantes pueden añadir productos al carro.
  \item \textit{Importancia}: Esencial.
  \item \textit{Validez}:
    \begin{itemize}
    \item \textit{Medible}: Junto a la descripción de cada producto, siempre deberá aparecer la opción de ``añadir al carrito'', que incluirá una unidad del producto seleccionado en el carro temporal.
    \item \textit{Alcanzable}: Se deben poder añadir elementos al carrito temporal que asocia el sistema a cada usuario.
    \item \textit{Relevante}: El hecho de mantener un carro o cesta supone un requisito básico sin el cual los usuarios no podrían comprar varios artículos de una vez, y tendrían que hacer una compra por artículo.
    \end{itemize}
  \end{itemize}


\item 
  \begin{itemize}
  \item \textit{Descripción}: Los visitantes pueden ver el contenido actual del carro.
  \item \textit{Importancia}: Esencial.
  \item \textit{Validez}:
    \begin{itemize}
    \item \textit{Medible}: El sistema deberá ofrecer un listado de todos los artículos que contiene el carro asociado a la sesión.
    \item \textit{Alcanzable}: El sistema debe mantener un carrito temporal asociado a la sesión del usuario, en el que se incluyan los productos que el usuario va añadiendo al mismo.
    \item \textit{Relevante}: El hecho de mantener un carro o cesta supone un requisito básico. Los usuarios deben poder visualizar el contenido del carro por si se da el caso de que quieran modificarlo antes de la compra.
    \end{itemize}
  \end{itemize}

\item 
  \begin{itemize}
  \item \textit{Descripción}: Los visitantes pueden eliminar productos del carro.
  \item \textit{Importancia}: Esencial.
  \item \textit{Validez}:
    \begin{itemize}
    \item \textit{Medible}: Junto a la descripción de cada artículo incluido en el carro, deberá estar disponible la opción de ``eliminar'' del carro.
    \item \textit{Alcanzable}: El carro asociado a la sesión que mantiene el sistema permitirá la eliminación de sus elementos.
    \item \textit{Relevante}: El hecho de mantener un carro o cesta supone un requisito básico. Si los usuarios no pueden suprimir elementos de su carro estarían obligados a no cometer errores a la hora de añadir productos a su compra.
    \end{itemize}
  \end{itemize}



\item 
  \begin{itemize}
  \item \textit{Descripción}: Los visitantes pueden buscar artículos.
  \item \textit{Importancia}: Condicional.
  \item \textit{Validez}:
    \begin{itemize}
    \item \textit{Medible}: La web ofrecerá un listado de artículos coincidentes con la búsqueda de una o varias palabras clave, mediante una casilla de búsqueda.
    \item \textit{Alcanzable}: El sistema permitirá la búsqueda de artículos en base a ciertos filtros.
    \item \textit{Relevante}: Sería bastante recomendable, aunque no esencial, ofrecer la posibilidad de buscar productos. Es una forma muy rápida de encontrar stock asociado a las necesidades del usuario.
    \end{itemize}
  \end{itemize}

\item 
  \begin{itemize}
  \item \textit{Descripción}: Los visitantes pueden recuperar su contraseña, en caso de que tengan una cuenta y no la recuerden.
  \item \textit{Importancia}: Esencial.
  \item \textit{Validez}:
    \begin{itemize}
    \item \textit{Medible}: Junto al formulario de login, aparecerá la opción de ``recordar contraseña''. El sistema, dado un nombre de usuario, enviará un correo electrónico al e-mail asociado a dicho nombre de usuario, indicándole la contraseña.
    \item \textit{Alcanzable}: El sistema enviará un email al usuario con su password asociado.
    \item \textit{Relevante}: Resulta bastante importante la posibilidad de que el sistema te recuerde tu contraseña, ya que, actualmente, con tantas redes sociales, webs de compras, etc. los usuarios suelen tener numerosas contraseñas, y frecuéntemente suelen ser olvidadas.
    \end{itemize}
  \end{itemize}

\item 
  \begin{itemize}
  \item \textit{Descripción}: Los visitantes pueden ver críticas sobre los productos de la tienda.
  \item \textit{Importancia}: Condicional.
  \item \textit{Validez}:
    \begin{itemize}
    \item \textit{Medible}: En la visualización de cada producto de la tienda se mostrarán las opiniones y críticas hechas por los usuarios de la web.
    \item \textit{Alcanzable}: Cada producto tendrá asociada una lista de opiniones hechas por los usuarios de la tienda.
    \item \textit{Relevante}: Es muy aconsejable ofrecer opiniones sobre los productos, para evitar posibles decepciones por parte del comprador potencial.
    \end{itemize}
  \end{itemize}

\item 
  \begin{itemize}
  \item \textit{Descripción}: Los clientes pueden finalizar una compra, pasando al proceso de pago.
  \item \textit{Importancia}: Esencial.
  \item \textit{Validez}:
    \begin{itemize}
    \item \textit{Medible}: Mientras se muestra el carro de la compra, el usuario debe tener disponible la opción de ``finalizar compra'', que cerrará el carro, impidiendo que se añadan o eliminen elementos del mismo.
    \item \textit{Alcanzable}: El carrito asociado a la sesión permitirá marcarlo como finalizado, para evitar el borrado o adición de productos.
    \item \textit{Relevante}: Es un requisito fundamental para poder completar un proceso de compra de forma satisfactoria.
    \end{itemize}
  \end{itemize}

\item 
  \begin{itemize}
  \item \textit{Descripción}: Los clientes pueden seleccionar la forma de pago, después de finalizar una compra.
  \item \textit{Importancia}: Esencial.
  \item \textit{Validez}:
    \begin{itemize}
    \item \textit{Medible}: Una vez se haya cerrado el carro, el cliente podrá seleccionar una forma de pago (tarjeta, PayPal, etc).
    \item \textit{Alcanzable}: El sistema ofrecerá un listado de formas de pago disponibles.
    \item \textit{Relevante}: Es un requisito fundamental para poder completar un proceso de compra de forma satisfactoria.
    \end{itemize}
  \end{itemize}

\item 
  \begin{itemize}
  \item \textit{Descripción}: Los clientes identificados pueden confirmar un pedido, que será el último paso del proceso de compra (tras seleccionar la forma de pago).
  \item \textit{Importancia}: Esencial.
  \item \textit{Validez}:
    \begin{itemize}
    \item \textit{Medible}: Tras seleccionar la forma de pago, el cliente podrá dar por finalizada la compra.
    \item \textit{Alcanzable}: El sistema permitirá finalizar un proceso de compra, una vez se haya finalizado el proceso de pago satisfactoriamente.
    \item \textit{Relevante}: Es un requisito fundamental para poder completar un proceso de compra de forma satisfactoria.
    \end{itemize}
  \end{itemize}

\item 
  \begin{itemize}
  \item \textit{Descripción}: Los clientes pueden modificar sus datos de usuario.
  \item \textit{Importancia}: Esencial.
  \item \textit{Validez}:
    \begin{itemize}
    \item \textit{Medible}: El cliente tendrá a su disposición un formulario para modificar sus datos de la cuenta (todos menos el nombre de usuario). El sistema modificará internamente la información de la cuenta del usuario.
    \item \textit{Alcanzable}: El sistema debe permitir la modificación de los clientes.
    \item \textit{Relevante}: Es bastante recomendable permitir que los usuarios modifiquen sus datos de cuenta, ya que así pueden corregir errores, modificar su email, etc.
    \end{itemize}
  \end{itemize}

\item 
  \begin{itemize}
  \item \textit{Descripción}: Los clientes pueden darse de baja, produciendo el borrado de su cuenta.
  \item \textit{Importancia}: Esencial.
  \item \textit{Validez}:
    \begin{itemize}
    \item \textit{Medible}: El cliente tendrá la posibilidad de eliminar su cuenta desde el formulario de modificación.
    \item \textit{Alcanzable}: El sistema tiene que permitir dar de baja a usuarios.
    \item \textit{Relevante}: Es bastante aconsejable permitir la eliminación de usuarios por parte de ellos mismos. En caso contrario el sistema almacenaría usuarios inútiles, o alguien tendría que encargarse de ir borrándolos uno a uno, lo cual sería absurdo.
    \end{itemize}
  \end{itemize}

\item 
  \begin{itemize}
  \item \textit{Descripción}: Los clientes pueden ver su historial de compras.
  \item \textit{Importancia}: Condicional.
  \item \textit{Validez}:
    \begin{itemize}
    \item \textit{Medible}: En la web se mostrará una sección con una tabla de historial de compras.
    \item \textit{Alcanzable}: El sistema debe poder recuperar un listado con todas las compras que ha hecho un usuario desde que creó su cuenta.
    \item \textit{Relevante}: No es esencial, pero sí recomendable para facilitar el trabajo del subsistema de recomendación.
    \end{itemize}
  \end{itemize}

\item 
  \begin{itemize}
  \item \textit{Descripción}: Los clientes pueden añadir un artículo a favoritos.
  \item \textit{Importancia}: Condicional.
  \item \textit{Validez}:
    \begin{itemize}
    \item \textit{Medible}: Junto a la descripción de cada artículo del stock, o de los artículos recomendados, la web mostrará la opción de añadir dicho artículo a los favoritos (siempre que el usuario se haya identificado).
    \item \textit{Alcanzable}: Cada cliente tiene asociada una lista de artículos favoritos, así como la posibilidad de añadir y eliminar productos de la misma.
    \item \textit{Relevante}: Se trata de un requisito fundamental para poder hacer uso del listado de favoritos.
    \end{itemize}
  \end{itemize}

\item 
  \begin{itemize}
  \item \textit{Descripción}: Los clientes pueden borrar un artículo de la lista de favoritos.
  \item \textit{Importancia}: Condicional.
  \item \textit{Validez}:
    \begin{itemize}
    \item \textit{Medible}: En la lista de artículos favoritos, junto a cada artículo, debe aparecer la opción de eliminar dicho artículo de la lista.
    \item \textit{Alcanzable}: Cada cliente tiene que permitir la eliminación de uno de sus artículos favoritos.
    \item \textit{Relevante}: Se trata de un requisito fundamental para poder hacer uso del listado de favoritos.
    \end{itemize}
  \end{itemize}

\item 
  \begin{itemize}
  \item \textit{Descripción}: Los clientes pueden ver el listado de sus artículos favoritos.
  \item \textit{Importancia}: Condicional.
  \item \textit{Validez}:
    \begin{itemize}
    \item \textit{Medible}: La web mostrará, para cada cliente, una sección con sus artículos marcados como favoritos.
    \item \textit{Alcanzable}: Cada cliente tiene asociada una lista de artículos favoritos, así como la posibilidad de añadir y eliminar productos de la misma.
    \item \textit{Relevante}: Se trata de un requisito fundamental para poder hacer uso del listado de favoritos.
    \end{itemize}
  \end{itemize}

\item 
  \begin{itemize}
  \item \textit{Descripción}: Los clientes pueden suscribirse al newsletter.
  \item \textit{Importancia}: Condicional.
  \item \textit{Validez}:
    \begin{itemize}
    \item \textit{Medible}: El formulario de modificación de cuenta muestra una opción de suscripción al newsletter.
    \item \textit{Alcanzable}: Cada cliente deberá tener asociada una propiedad booleana de suscripción al newsletter.
    \item \textit{Relevante}: Es un requisito recomendable para la promoción de artículos.
    \end{itemize}
  \end{itemize}

\item 
  \begin{itemize}
  \item \textit{Descripción}: Los clientes pueden borrarse del newsletter.
  \item \textit{Importancia}: Condicional.
  \item \textit{Validez}: 
    \begin{itemize}
    \item \textit{Medible}: El formulario de modificación de cuenta muestra una opción de suscripción al newsletter.
    \item \textit{Alcanzable}: Cada cliente deberá tener asociada una propiedad booleana de suscripción al newsletter.
    \item \textit{Relevante}: Requisito fundamental para evitar que los usuarios se cansen de recibir publicidad de la tienda.
    \end{itemize}
  \end{itemize}


\item 
  \begin{itemize}
  \item \textit{Descripción}: Los clientes pueden cancelar un pedido completado (dentro del margen de tiempo).
  \item \textit{Importancia}: Condicional.
  \item \textit{Validez}:
    \begin{itemize}
    \item \textit{Medible}: En el historial de pedidos, el cliente debe poder cancelar cualquier pedido antes de que pasen 24 horas desde que se finalizó.
    \item \textit{Alcanzable}: Los pedidos deben tener un estado modificable durante las primeras 24 horas. Hasta pasado ese tiempo el pedido no se procesaría realmente.
    \item \textit{Relevante}: Es un requisito recomendable por si un cliente se arrepiente, por cualquier razón, de una compra ya finalizada.
    \end{itemize}
  \end{itemize}

\item 
  \begin{itemize}
  \item \textit{Descripción}: Los clientes pueden escribir críticas de los productos, siempre y cuando los hayan comprado antes.
  \item \textit{Importancia}: Condicional.
  \item \textit{Validez}:
    \begin{itemize}
    \item \textit{Medible}: Junto a la descripción de cada artículo, el cliente tendrá la opción de escribir una reseña del mismo. Dicha posibilidad sólo podrá ser utilizada si el cliente compró alguna vez dicho artículo.
    \item \textit{Alcanzable}: Cada producto tiene asociada una lista de opiniones, en las que estaría indicado el usuario autor de dicha crítica.
    \item \textit{Relevante}: Este requisito es de relativa importancia, clave para el funcionamiento del sistema de críticas y opiniones. 
    \end{itemize}
  \end{itemize}

\item 
  \begin{itemize}
  \item \textit{Descripción}: Los clientes pueden cerrar su sesión.
  \item \textit{Importancia}: Esencial.
  \item \textit{Validez}:
    \begin{itemize}
    \item \textit{Medible}: El sistema pondrá a disposición del cliente identificado una opción para salir de la sesión.
    \item \textit{Alcanzable}: La sesión asociada a un cliente identificado puede ser cerrada. Al salir de la sesión, el usuario volverá a ser considerado visitante.
    \item \textit{Relevante}: Se trata de un requisito fundamental para el buen funcionamiento de la plataforma web. Por cuestiones de seguridad sobre todo, los usuarios deben poder salir de su sesión.
    \end{itemize}
  \end{itemize}

\item 
  \begin{itemize}
  \item \textit{Descripción}: Los clientes pueden consultar el estado de sus pedidos.
  \item \textit{Importancia}: Condicional.
  \item \textit{Validez}:
    \begin{itemize}
    \item \textit{Medible}: En el historial de los pedidos de un cliente se mostrará el estado de cada uno de ellos.
    \item \textit{Alcanzable}: El sistema debe poder recuperar un listado con todas las compras que ha hecho un usuario desde que creó su cuenta. En ese listado debe aparecer el estado de los pedidos.
    \item \textit{Relevante}: Es un requisito interesante para mejorar la calidad de la tienda online. Viendo el estado de un pedido, los clientes pueden saber dónde se encuentra su envío, o si aún tienen la posibilidad de cancelarlo.
    \end{itemize}
  \end{itemize}

\item 
  \begin{itemize}
  \item \textit{Descripción}: Los administradores pueden dar de alta un artículo.
  \item \textit{Importancia}: Esencial.
  \item \textit{Validez}:
    \begin{itemize}
    \item \textit{Medible}: Un usuario autenticado como administrador tiene que poder añadir productos al stock.
    \item \textit{Alcanzable}: El sistema debe permitir que un usuario con permisos de administración cree artículos nuevos.
    \item \textit{Relevante}: Requisito fundamental para el mantenimiento del stock de la tienda.
    \end{itemize}
  \end{itemize}

\item 
  \begin{itemize}
  \item \textit{Descripción}: Los administradores pueden dar de baja un artículo.
  \item \textit{Importancia}: Esencial.
  \item \textit{Validez}:
    \begin{itemize}
    \item \textit{Medible}: Un usuario autenticado como administrador tiene que poder eliminar productos del stock.
    \item \textit{Alcanzable}: El sistema debe permitir que un usuario con permisos de administración elimine artículos existentes.
    \item \textit{Relevante}: Requisito fundamental para el mantenimiento del stock de la tienda.
    \end{itemize}
  \end{itemize}

\item 
  \begin{itemize}
  \item \textit{Descripción}: Los administradores pueden modificar un artículo.
  \item \textit{Importancia}: Esencial.
  \item \textit{Validez}:
    \begin{itemize}
    \item \textit{Medible}: Un usuario autenticado como administrador tiene que poder cambiar la información de los productos del stock.
    \item \textit{Alcanzable}: El sistema debe permitir que un usuario con permisos de administración modifique un producto existente.
    \item \textit{Relevante}: Requisito fundamental para el mantenimiento del stock de la tienda.
    \end{itemize}
  \end{itemize}


\end{enumerate}



\section{Requisitos de evolución}

\begin{enumerate}[{\bf EV-1}]
\item 
  \begin{itemize}
  \item \textit{Descripción}: La aplicación web debe mantenerse siempre en la última versión disponible. Mientras sea posible, los cambios deben ser trasparentes al el usuario.
  \item \textit{Importancia}: Condicional
  \item \textit{Validez}:

    \begin{itemize}
    \item \textit{Alcanzable}: Para cumplir este requisito, se debe mantener un control de versiones actualizado, y trabajar sobre él de forma regular.
    \item \textit{Relevante}: Es importante mantener bien actualizado el software. Aunque funcione bien, deberá recibir mejoras periódicamente.
    \end{itemize}

  \end{itemize}

\end{enumerate}

\section{Requisitos de soporte}

\begin{enumerate}[{\bf SO-1}]
\item 
  \begin{itemize}
  \item \textit{Descripción}: El usuario final de la aplicación web debe contar con una conexión a Internet totalmente funcional para poder trabajar con el sistema.
  \item \textit{Importancia}: Esencial
  \end{itemize}

\item 
  \begin{itemize}
  \item \textit{Descripción}: El usuario debe tener un navegador web funcionando en su computador, con efecto de poder acceder al sistema web.
  \item \textit{Importancia}: Esencial
  \end{itemize}

\end{enumerate}


\chapter{Fase de elaboración}\label{CAPElaboracion}

\section{Arquitectura general}

La arquitectura general del sistema se muestra en el diagrama de despliegue de la figura \ref{despliegue}.

\imagenanchototal{./diagramas/Despliegue.png}{Diagrama de despliegue del sistema}{despliegue}

\section{Casos de uso}

Partiendo de los requisitos funcionales especificados en el capítulo anterior, planteamos los siguientes casos de uso:

\notamargen{Diagrama de casos de uso}{El diagrama completo de casos de uso puede verse en el anexo \ref{casosuso}}

\begin{enumerate}[{\bf UC-1}]

\item 
  \begin{itemize}
  \item {\it Caso de uso}: Identificación.
  \item {\it Descripción}: Un visitante inicia su sesión como cliente o administrador.
  \item {\it Precondiciones}:
    \begin{itemize}
    \item El usuario debe estar registrado.
    \end{itemize}
  \item {\it Actores}: Visitante, Administrador.  
  
  
  \item {\it Flujo normal}: autenticación exitosa
    \begin{itemize}
    \item Secuencia de eventos:
      \begin{enumerate}[1. ]
      \item El usuario introduce su nombre de usuario y contraseña en el formulario web.
      \item La interfaz envía los datos al módulo de identificación.
      \item El módulo gestor de usuarios solicita los datos completos del usuario al módulo de persistencia.
      \item El módulo de persistencia devuelve la información del usuario al módulo gestor de usuarios.
      \item El módulo gestor de usuarios comprueba los datos y le entrega a la interfaz la información del usuario para que esta cambie acorde con él.
      \item La web cambia para dar la bienvenida al usuario y mostrarle las opciones propias de ese usuario (cliente autenticado o administrador).
      \end{enumerate}
    
    \item {\it Postcondiciones}: El visitante pasa a ser un cliente identificado o administrador.  La web cambia para dar la bienvenida al usuario y mostrarle las opciones propias de ese usuario (cliente autenticado o administrador).
    \end{itemize}

    \imagenanchototal{diagramas/diagramassecuencia/uc-login.png}{Caso de Uso Identificación Escenario 1}{casousologinescenario1}
    
  \item {\it Flujo alternativo 1}: autenticación errónea
  \begin{itemize}
    \item Secuencia de eventos:
      
  
    \begin{enumerate}[1. ]
    \item El usuario introduce su nombre de usuario y contraseña en el formulario web.
    \item La interfaz envía los datos al módulo gestor de usuarios.
    \item El módulo gestor de usuarios solicita los datos completos del usuario al módulo de persistencia.
    \item El módulo de persistencia devuelve la información del usuario al módulo gestor de usuarios.
    \item El módulo gestor de usuarios comprueba los datos y, como no son correctos porque la combinación \textit{login}-\textit{password} no existe, lanza un error a la capa de presentación.
    \item La web muestra un error de autenticación.
    \end{enumerate}
  \item {\it Postcondiciones}: La web muestra un error de autenticación. 
\end{itemize}

    \imagenanchototalimxpar{diagramas/diagramassecuencia/uc-login-alternativo1.png}{Caso de Uso Identificación Escenario Alternativo}{casousologinescenarioalternativo}

  \end{itemize}

\item 
  \begin{itemize}
  \item {\it Caso de uso}: Registrarse.
  \item {\it Descripción}: Un visitante crea una cuenta en el sistema para poder hacer compras, tener un historial de sus pedidos, mantener una lista de productos favoritos, etc.
  \item {\it Actores}: Visitante, Sistema.
  \item {\it Precondiciones}: ninguna


  \item {\it Flujo normal}: registro exitoso.
    \begin{itemize}
\item Secuencia de eventos:
    \begin{enumerate}[1. ]
    \item El usuario introduce los datos mínimos de registro (nombre de usuario, email y contraseña) en la web.
    \item La interfaz gráfica web envía los datos al módulo gestor de usuarios.
    \item El módulo gestor de usuarios solicita la creación de la cuenta correspondiente al módulo de persistencia.
    \item El módulo de persistencia devuelve una confirmación de creación de la cuenta.
    \item El módulo gestor de usuarios devuelve la confirmación a la interfaz web.
    \item La interfaz web muestra la confirmación de que la cuenta se ha creado satisfactoriamente.
    \end{enumerate}
  \item {\it Postcondiciones}: La interfaz web muestra la confirmación de que la cuenta se ha creado satisfactoriamente. El visitante pasa a ser un cliente identificado.
    \end{itemize}
    \imagenanchototalimpar{diagramas/diagramassecuencia/uc-registro.png}{Caso de Uso Registro Escenario 1}{casousoregistroescenario1}

  \item {\it Flujo alternativo 1}: registro fallido (nombre de usuario existente).
\begin{itemize}
\item Secuencia de eventos:

    \begin{enumerate}[1. ]
    \item El usuario introduce los datos mínimos de registro (nombre de usuario, email y contraseña) en la web.
    \item La interfaz gráfica web envía los datos al módulo gestor de usuarios.
    \item El módulo gestor de usuarios solicita la creación de la cuenta correspondiente al módulo de persistencia.
    \item El módulo de persistencia devuelve un error indicando que el nombre de usuario ya existe.
    \item El módulo gestor de usarios devuelve el error a la interfaz web.
    \item La interfaz web muestra el error al usuario, solicitándole un nombre de usuario diferente.
    \end{enumerate}
  \item {\it Postcondiciones}: La interfaz web muestra el error al usuario, solicitándole un nombre de usuario diferente. \end{itemize}

    \imagenanchototal{diagramas/diagramassecuencia/uc-registro-alternativo1.png}{Caso de Uso Registro Escenario Alternativo}{casousoregistroescenarioalternativo}

  \item {\it Flujo alternativo 2}: registro fallido (email ya asignado a una cuenta existente).
\begin{itemize}
\item Secuencia de eventos:

    \begin{enumerate}[1. ]
    \item El usuario introduce los datos mínimos de registro (nombre de usuario, email y contraseña) en la web.
    \item La interfaz gráfica web envía los datos al módulo gestor de usuarios.
    \item El módulo gestor de usuarios solicita la creación de la cuenta correspondiente al módulo de persistencia.
    \item El módulo de persistencia devuelve un error indicando que el email indicado ya está asociado a una cuenta de usuario.
    \item El módulo gestor de usuarios devuelve el error a la interfaz web.
    \item La interfaz web muestra el error al usuario, recomendándole que recupere su contraseña si no la recuerda.
    \end{enumerate}
  \item {\it Postcondiciones}: La interfaz web muestra el error al usuario, recomendándole que recupere su contraseña si no la recuerda. 
\end{itemize}

    \imagenanchototal{diagramas/diagramassecuencia/uc-registro-alternativo2.png}{Caso de Uso Registro Escenario Alternativo 2}{casousoregistroescenarioalternativo2}

  \end{itemize}

\item 
  \begin{itemize}
  \item {\it Caso de uso}: Obtener Productos.
  \item {\it Descripción}: El sistema obtiene los productos que posee cuando el usuario entra a la web.
  \item {\it Precondiciones}: Ninguna
  \item {\it Actores}: Sistema de gestión de base de datos


  \item {\it Flujo normal}: Muestra de productos en stock
\begin{itemize}
\item Secuencia de eventos:
    \begin{enumerate}[1. ]
    \item El sistema pide al módulo gestor de productos los datos de los productos en stock.
    \item El módulo gestor de productos pide al módulo de persistencia de productos que tiene en stock.
    \item El módulo de persistencia devuelve una lista de productos al módulo gestor de productos.
    \item El módulo gestor de productos devuelve una lista de productos al sistema.
    \item El sistema le comunica a la interfaz gráfica la lista de productos que debe mostrar.
    \item La interfaz gráfica se reacarga para mostrar la información de los productos.
    \end{enumerate}
  \item {\it Postcondiciones}: La interfaz gráfica se reacarga para mostrar la información de los productos.
\end{itemize}
        \imagenanchototalimpar{diagramas/diagramassecuencia/uc-obtener.png}{Caso de Uso Obtener Producto Escenario}{casousoobtenerescenario}

  \item {\it Flujo alternativo 1}: No hay productos que mostrar
\begin{itemize}
\item Secuencia de eventos:

    \begin{enumerate}[1. ]
    \item El sistema pide al módulo gestor de productos los datos de los productos en stock.
    \item El módulo gestor de productos pide al módulo de persistencia de productos que tiene en stock.
    \item El módulo de persistencia devuelve una lista de productos vacía al módulo gestor de productos.
    \item El módulo gestor de productos devuelve una lista de productos vacía al sistema.
    \item El sistema le comunica a la interfaz gráfica que debe mostrar ``No disponemos de productos en stock.''.
    \item La interfaz gráfica se reacarga para mostrar la información.
    \end{enumerate}
  \item {\it Postcondiciones}: La interfaz gráfica se reacarga para mostrar ``No disponemos de productos en stock.''.
  \end{itemize}

        \imagenanchototalpar{diagramas/diagramassecuencia/uc-obtener-alternativo1.png}{Caso de Uso Obtener Productos Escenario Alternativo}{casousoobteneralternativo}

  \end{itemize}

\item 
  \begin{itemize}
  \item {\it Caso de uso}: Modificar cuenta
  \item {\it Descripción}: Un cliente identificado modifica los datos personales de su cuenta.
  \item {\it Precondiciones}: La web muestra al cliente el mensaje de cambio afirmativo.
    \begin{itemize}
    \item Debe estar identificado como cliente.
    \end{itemize}
  \item {\it Actores}: Cliente identificado.


  \item {\it Flujo normal}: cambio de datos existoso.
    \begin{itemize}
\item Secuencia de eventos:
    \begin{enumerate}[1. ]
    \item El cliente accede a la intefaz de gestión de datos personales.
    \item	El cliente realiza los cambios deseados e introduce su contraseña actual.
    \item La interfaz envía los nuevos datos al módulo gestor de usuarios.
    \item El módulo gestor de usuarios envía los datos al módulo de persistencia.
    \item El módulo de persistencia sobreescribe los antiguos datos del cliente con los nuevos y devuelve la confirmación al gestor.
    \item El módulo de gestor de usuarios manda un mensaje de confirmación a la interfaz.
    \item La web muestra al cliente el mensaje de cambio afirmativo.
    \end{enumerate}
  \item {\it Postcondiciones}: La web se recarga y muestra el mensaje de cambio realizado correctamente.
\end{itemize}

    \imagenanchototal{diagramas/diagramassecuencia/uc-modificar_cuenta.png}{Caso de Uso Registro Modificar Cuenta Escenario Normal}{casousomodificar}

  \item {\it Flujo alternativo 1}: Cambio rechazado.
\begin{itemize}
\item Secuencia de eventos:

    \begin{enumerate}[1. ]
	\item El cliente accede a la intefaz de gestión de datos personales.
    \item	El cliente realiza los cambios deseados e introduce su contraseña actual.
    \item El módulo gestor de usuarios envía los datos al módulo de persistencia.
    \item El módulo de persistencia detecta un error en los cambios realizados y devuelve un error al gestor.
    \item El módulo gestor de usuario manda un mensaje de error a la interfaz.
    \item La web muestra un error en los cambios.
    \end{enumerate}
  \item {\it Postcondiciones}:  La web muestra un mensaje de error en los cambios.
\end{itemize}
    
    \imagenanchototalimpar{diagramas/diagramassecuencia/uc-modificar_cuenta-alternativo1.png}{Caso de Uso Modificar Cuenta Escenario Alternativo}{casousomodificarescenarioalternativo}

   \item {\it Flujo alternativo 2}: Darse de baja.
\begin{itemize}
\item Secuencia de eventos:

   
    \begin{enumerate}[1. ]
	\item El cliente accede a la intefaz de gestión de datos personales.
    \item	El cliente selecciona la opción de darse de baja e introduce su contraseña.
    \item La interfaz envía la petición al módulo gestor de usuarios.
    \item El módulo gestor de usuarios deslogea y envía una orden al módulo de persistencia para que destruya todos los datos del cliente.
    \item La web muestra su página principal.
    \end{enumerate}
 \item {\it Postcondiciones}: La web muestra su página principal. El cliente identificado pasa a ser un visitante.
\end{itemize}

    \imagenanchototalpar{diagramas/diagramassecuencia/uc-modificar_cuenta-alternativo1.png}{Caso de Uso Modificar Cuenta Escenario Alternativo 2}{casousomodificarescenarioalternativo2}

  \end{itemize}

\item 
  \begin{itemize}
  \item {\it Caso de uso}: Ver Carro
  \item {\it Descripción}: Un cliente identificado o visitante comprueba los productos que tiene en el pedido que está realizando.
  \item {\it Precondiciones}: Ninguna


  \item {\it Actores}: Cliente identificado, visitante.
  \item {\it Flujo normal}: mostrar productos del carro.
\begin{itemize}
\item Secuencia de eventos:
    \begin{enumerate}[1. ]
    \item El cliente o visitante accede a la intefaz del carro.
    \item La web muestra los productos que tiene el carro, el número de artículos y su precio.
      


    \end{enumerate}
  \item {\it Postcondiciones}: La web muestra los productos que tiene el carro, el número de artículos y su precio.
\end{itemize}

      \imagenanchototal{diagramas/diagramassecuencia/uc-ver_carro.png}{Caso de Uso Ver Carro Escenario Normal}{casousovercarroescenarionormal}

  \item {\it Flujo alternativo 1}: Quitar producto.
\begin{itemize}
\item Secuencia de eventos:

    \begin{enumerate}[1. ]
	\item El cliente o visitante accede a la intefaz del carro.
    \item La web muestra los productos que tiene el carro, el número de artículos y su precio.
    \item El cliente o visitante elimina un producto.
    \item La interfaz muestra los productos restantes que contiene el carro.
    \end{enumerate}
  \item {\it Postcondiciones}: La interfaz muestra los productos restantes que contiene el carro.
\end{itemize}
    
    \imagenanchototalimpar{diagramas/diagramassecuencia/uc-ver_carro-alternativo1.png}{Caso de Uso Ver Carro Escenario Alternativo}{casousovercarroescenarioalternativo}    

   \item {\it Flujo alternativo 2}: Añadir al carro.
\begin{itemize}
\item Secuencia de eventos:

    \begin{enumerate}[1. ]
	\item El cliente o visitante accede a la intefaz del carro.
    \item La web muestra los productos que tiene el carro, el número de artículos de ese producto y su precio.
    \item El cliente o visitante añade un producto que ya tenga en el carro incrementando el número de artículos.
    \item  La interfaz actualiza el carro.
    \end{enumerate}
   \item {\it Postcondiciones}: La interfaz muestra los productos que contiene el carro más las unidades del nuevo producto añadido.
\end{itemize}

    \imagenanchototalpar{diagramas/diagramassecuencia/uc-ver_carro-alternativo2.png}{Caso de Uso Ver Carro Escenario Alternativo2}{casousovercarroescenarioalternativo2}    

     \item {\it Flujo alternativo 3}: Finalizar compra.
\begin{itemize}
\item Secuencia de eventos:

    \begin{enumerate}[1. ]
	\item El cliente o visitante accede a la intefaz del carro.
    \item La web muestra los productos que tiene el carro, el número de artículos de ese producto y su precio.
    \item El cliente o visitante selecciona la opción de finalizar compra.
    \item La web cambia y muestra la interfaz de finalización de compra.
    \end{enumerate}
     \item {\it Postcondiciones}: La web cambia y muestra la interfaz de finalización de compra.
\end{itemize}

    \imagenanchototalimpar{diagramas/diagramassecuencia/uc-ver_carro-alternativo3.png}{Caso de Uso Ver Carro Escenario Alternativo3}{casousovercarroescenarioalternativo3}    

  \end{itemize}

\item 
  \begin{itemize}
  \item {\it Caso de uso}: Añadir producto.
  \item {\it Descripción}: Los administradores pueden dar de alta un artículo.
  \item {\it Precondiciones}: La web se refresca indicando que el producto se ha añadido correctamente.
    \begin{itemize}
    \item El administrador está identificado como tal.
    \end{itemize}

  \item {\it Actores}: Administrador.

  \item {\it Flujo normal}: articulo nuevo añadido correctamente.
\begin{itemize}
\item Secuencia de eventos:
    \begin{enumerate}[1. ]
    \item El administrador rellena los datos y stock del producto.
    \item La interfaz envía los datos al módulo gestor de productos.
    \item El módulo gestor de productos se comunica con el modulo de persistencia.
    \item Módulo de persistencia actualiza la BD con el nuevo producto.
    \item El módulo gestor de productos devuelve el mensaje de éxito a la interfaz.
    \item La web se refresca indicando que el producto se ha añadido correctamente.

    \end{enumerate}

  \item {\it Postcondiciones}: La web se refresca mostrando un mensaje de que el producto se ha añadido correctamente.
\end{itemize}

  \imagenanchototalpar{diagramas/diagramassecuencia/uc-anadir_Producto.png}{Caso de Añadir Producto Escenario Normal}{casousoanadirproductoescenarionormal}    

  \item {\it Flujo alternativo 1}: El articulo a añadir ya existe.
\begin{itemize}
\item Secuencia de eventos:

    \begin{enumerate}[1. ]
    \item El administrador rellena los datos y stock del producto.
    \item La interfaz envía los datos al módulo gestor de productos.
    \item El módulo gestor de productos se comunica con el modulo de persistencia.
    \item Módulo gestor de productos comprueba que el articulo ya existe.
    \item El módulo gestor de productos devuelve el mensaje de error a la interfaz.
    \item La web se refresca indicando que el producto ya existía, y por lo tanto sólo puede modificarlo  o borrarlo.

    \end{enumerate}
    \item {\it Postcondiciones}: La web se refresca indicando que el producto ya existe.
\end{itemize}

  \imagenanchototal{diagramas/diagramassecuencia/uc-anadir_Producto-alternativo1.png}{Caso de Añadir Producto Escenario Alternativo}{casousoanadirproductoescenarioalternativo}    

  \item {\it Flujo alternativo 2}: El articulo a añadir se intenta añadir con cero (o menos) de stock.
\begin{itemize}
\item Secuencia de eventos:

    \begin{enumerate}[1. ]
    \item El administrador rellena los datos e indica cero en el stock del producto.
    \item La interfaz envía los datos al módulo gestor de productos.
    \item Módulo gestor de productos comprueba que el stock no es mayor que cero, y envía el error a la interfaz.
    \item La web se refresca indicando que el producto no puede añadirse con un stock de cero o menos.


    \end{enumerate}
    \item {\it Postcondiciones}: La web se refresca indicando que el producto no puede añadirse con un stock de cero o menos.
\end{itemize}
    \imagenanchototalpar{diagramas/diagramassecuencia/uc-anadir_Producto-alternativo2.png}{Caso de Añadir Producto Escenario Alternativo 2}{casousoanadirproductoescenarioalternativo2}    

  \end{itemize}
\item 
  \begin{itemize}
  \item {\it Caso de uso}: Borrar producto.
  \item {\it Descripción}: Los administradores pueden dar de baja un artículo.
  \item {\it Precondiciones}: 
    \begin{itemize}
    \item El administrador está identificado como tal, y el articulo existe en el sistema.
    \end{itemize}
  
  \item {\it Actores}: Administrador.

  \item {\it Flujo normal}: Artículo existente borrado correctamente.
\begin{itemize}
\item Secuencia de eventos:
    \begin{enumerate}[1. ]
    \item El administrador introduce en la interfaz el artículo a buscar.
    \item La interfaz envía los datos al módulo gestor de productos.
    \item El módulo gestor de productos se comunica con el modulo de persistencia.
    \item Módulo de persistencia le devuelve al gestor de productos los datos el producto buscado.
    \item El módulo gestor de productos devuelve los datos del producto a la interfaz.
    \item La web se refresca mostrando el producto buscado.
    \item El administrador pulsa el botón de borrar el producto.
    \item La interfaz envía la petición de borrado al módulo gestor de productos.
    \item El módulo gestor de productos se comunica con el módulo de persistencia.
    \item Módulo de persistencia borra del sistema ese producto (y su stock), y devuelve el exito de la acción al gestor de productos.
    \item El módulo gestor de productos devuelve los datos de éxito de borrado a la interfaz.
    \item La web se refresca mostrando que el producto se ha borrado.
    \end{enumerate}
\item {\it Postcondiciones}: La web se refresca mostrando que el producto se ha borrado.
\end{itemize}

    \imagenanchototalimpar{diagramas/diagramassecuencia/uc-borrar_articulo.png}{Caso de Borrar Artículo Escenario Normal}{casousoborrararticuloescenarionormal}    

  \item {\it Flujo Alternativo}: Artículo no existente.
\begin{itemize}
\item Secuencia de eventos:
    \begin{enumerate}[1. ]
    \item El administrador introduce en la interfaz el artículo a buscar.
    \item La interfaz envía los datos al módulo gestor de productos.
    \item El módulo gestor de productos se comunica con el modulo de persistencia.
    \item Módulo de persistencia le devuelve al gestor de productos la lista de productos buscados vacía.
    \item El módulo gestor de productos devuelve la lista vacía de los datos del producto a la interfaz.
    \item La web se refresca mostrando el mensaje ``No se ha encontrado el artículo.''
    \end{enumerate}
    \item {\it Postcondiciones}: La web se refresca mostrando el mensaje ``No se ha encontrado el artículo.''
\end{itemize}

    \imagenanchototalpar{diagramas/diagramassecuencia/uc-borrar_articulo-alternativo1.png}{Caso de Borrar Artículo Escenario Alternativo}{casousoborrararticuloescenarioalternativo}    

  \end{itemize}

\item 
  \begin{itemize}
  \item {\it Caso de uso}: Modificar producto.
  \item {\it Descripción}: Los administradores pueden modificar un artículo.
  \item {\it Precondiciones}: 
    \begin{itemize}
    \item El administrador está identificado como tal, y el articulo existe en el sistema.
    \end{itemize}


  \item {\it Actores}: Administrador.

  \item {\it Flujo normal}: Artículo se modifica correctamente.
\begin{itemize}
\item Secuencia de eventos:
    \begin{enumerate}[1. ]
    \item El administrador introduce en la interfaz el artículo a buscar.
    \item La interfaz envía los datos al módulo gestor de productos.
    \item El módulo gestor de productos se comunica con el modulo de persistencia.
    \item Módulo de persistencia le devuelve al gestor de productos los datos el producto buscado.
    \item El módulo gestor de productos devuelve los datos del producto a la interfaz.
    \item La web se refresca mostrando el producto buscado.
    \item El administrador pulsa el botón de modificar el producto.
    \item La interfaz envía la petición de modificación al módulo gestor de productos.
    %\item El módulo gestor de productos se comunica con el módulo de persistencia.
    \item El módulo gestor de productos devuelve los datos del producto a la interfaz.
    \item La web se refresca mostrando los datos  del producto de manera que se puedan editar.
    \item El administrador cambia los campos del producto que considere oportuno (stock, nombre, categoría, descripción, etc), y pulsa el botón de aceptar cambio.
    \item La interfaz envía los nuevos datos al módulo gestor de productos.
    \item El módulo gestor de productos comprueba que los datos son válidos, y se comunica con el módulo de persistencia indicandole los nuevos datos.
    \item El módulo gestor de persistencia comunica al gestor de productos que la acción se ha realizado con éxito.
    \item El módulo gestor de productos devuelve a la interfaz que los datos ha sido modificados con éxito.
    \item La web se refresca mostrando que el producto ha sido modificado.
    \end{enumerate}
  \item {\it Postcondiciones}: La web se refresca mostrando que el producto ha sido modificado.
\end{itemize}

  \item {\it Flujo alternativo}: Artículo se modifica incorrectamente.

\begin{itemize}
\item Secuencia de eventos:   

    \begin{enumerate}[1. ]
    \item El administrador introduce en la interfaz el artículo a buscar.
    \item La interfaz envía los datos al módulo gestor de productos.
    \item El módulo gestor de productos se comunica con el modulo de persistencia.
    \item Módulo de persistencia le devuelve al gestor de productos los datos el producto buscado.
    \item El módulo gestor de productos devuelve los datos del producto a la interfaz.
    \item La web se refresca mostrando el producto buscado.
    \item El administrador pulsa el botón de modificar el producto.
    \item La interfaz envía la petición de modificación al módulo gestor de productos.
    %\item El módulo gestor de productos se comunica con el módulo de persistencia.
    \item El módulo gestor de productos devuelve los datos del producto a la interfaz.
    \item La web se refresca mostrando los datos  del producto de manera que se puedan editar.
    \item El administrador cambia el stock a un valor menor o igual que cero (y puede que otros campos) y pulsa el botón de aceptar cambio.
    \item La interfaz envía los nuevos datos al módulo gestor de productos.
    \item El módulo gestor de productos comprueba que los datos no son válidos (el stock es menor o igual a cero).
    \item El módulo gestor de productos devuelve a la interfaz que los datos no han sido modificados.
    \item La web se refresca mostrando que el producto no ha sido modificado.
    \end{enumerate}
 \item {\it Postcondiciones}: La web se refresca mostrando que el producto no ha sido modificado.
  \end{itemize}
\end{itemize}

\end{enumerate}



\section{Casos de Pruebas}
Para los diferentes casos de uso especificados se definirán casos de prueba acordes.

% Obtener Productos
\begin{enumerate}
\item {\it Identificador:} Obtener Productos.EscenarioNormal.Casoprueba1
\item {\it Situaciones iniciales:}
  \begin{itemize}
  \item Hay 4 productos en la base de datos el producto 001 botas a 70 euros y hay 50 pares de unidades, el producto 002 pantalones a 30 euros y hay 40, el producto 003 camisas a 20 euros y hay 200 y el producto 004 gorras a 10 euros y hay 10.
  \end{itemize}
\item {\it Mensajes:}
  \begin{itemize}
  \item mostrar\_productos()
    \begin{itemize}
    \item {\it Resultado esperado:}
      \begin{itemize}
      \item Hay 4 productos.
        \begin{itemize}
        \item Hay 50 pares de botas.
        \item Hay 40 pantalones.
        \item Hay 20 camisas.
        \item Hay 10 gorras.
        \end{itemize}
      \end{itemize}
    \end{itemize}
  \end{itemize}
\end{enumerate}

\begin{enumerate} 
\item {\it Identificador:} Obtener Productos.EscenarioAlternativo.CasoPrueba2
\item {\it Situaciones iniciales:}
  \begin{itemize}
  \item No hay productos en la base de datos.
  \end{itemize}
\item {\it Mensajes:}
  \begin{itemize}
  \item mostrar\_productos()
    \begin{itemize}
    \item {\it Resultado esperado:}
      \begin{itemize}
      \item Mostrar: No existen productos en stock.
      \end{itemize}
    \end{itemize}
  \end{itemize}
\end{enumerate}


%Modificar Cuenta.

\begin{enumerate}
	\item {\it Identificador:} Modificar Cuenta.EscenarioNormal.CasoPrueba1
	\item {\it Situaciones iniciales:}
    		\begin{itemize}
    			\item Todos los datos correctos
		\end{itemize}
	\item {\it Mensajes:}
		\begin{itemize}
			\item modificarDatos(nombre,apellido,nick,email,password,telefono)
				 \begin{itemize}
					 	\item Resultado esperado:
					 \begin{itemize}
		       			 \item Mensaje de de confirmación: Datos modificados correctamente.
		       			 \item Cambios realizados en el sistema.
					 \end{itemize}
				 \end{itemize}
		\end{itemize}
\end{enumerate}

\begin{enumerate}
	\item {\it Identificador:} Modificar Cuenta.EscenarioAlternativo.CasoPrueba2
	\item {\it Situaciones iniciales:}
    		\begin{itemize}
    			\item Correo electrónico no valido (Fallo en el prototipo de un correo. Es decir, falta la \@, el dominio,\ldots)
		\end{itemize}
	\item {\it Mensajes:}
		\begin{itemize}
			\item modificarDatos(nombre,apellido,nick,email,password,telefono)
				\begin{itemize}
					 \item Resultado esperado:
					 \begin{itemize}
		       			 \item Mensaje de error: Correo electrónico inválido.
					 \end{itemize}
				\end{itemize}
		\end{itemize}
\end{enumerate}


\begin{enumerate}
	\item {\it Identificador:} Modificar Cuenta.EscenarioAlternativo.CasoPrueba3
	\item {\it Situaciones iniciales:} 
    		\begin{itemize}
    			\item El cliente intenta cambiar su nick a uno ya existente.
		\end{itemize}
	\item {\it Mensajes:}
		\begin{itemize}
			\item modificarDatos(nombre,apellido,nick,email,password,telefono)
				\begin{itemize}
					 \item Resultado esperado:
					 \begin{itemize}
		       			 \item Mensaje de error: El nombre de usuario ya existe. Escoja otro.
					 \end{itemize}
				\end{itemize}
		\end{itemize}
\end{enumerate}

\begin{enumerate}
	\item {\it Identificador:} Modificar Cuenta.EscenarioAlternativo.CasoPrueba4
	\item {\it Situaciones iniciales:} 
    		\begin{itemize}
    			\item Campos obligatorios vacios (El cliente no rellena campos obligatorios)
		\end{itemize}
	\item {\it Mensajes:}
		\begin{itemize}
			\item modificarDatos(nombre,apellido,nick,email,password,telefono)
				\begin{itemize}
					 \item Resultado esperado:
					 \begin{itemize}
		       			 \item Mensaje de error: Faltan campos por especificar.
					 \end{itemize}
				\end{itemize}
		\end{itemize}
\end{enumerate}


\begin{enumerate}
	\item {\it Identificador:} Modificar Cuenta.EscenarioNormal.CasoPrueba5
	\item {\it Situaciones iniciales:} 
    		\begin{itemize}
    			\item El usuario se da de baja.
		\end{itemize}
	\item {\it Mensajes:}
		\begin{itemize}
			\item darseBaja()
				\begin{itemize}
					 \item Resultado esperado:
					 \begin{itemize}
					 	 \item Se deslogea al cliente.
					 	 \item Se destruyen los datos del cliente.
		       			 \item Se muestra la página principal.
					 \end{itemize}
				\end{itemize}
		\end{itemize}
\end{enumerate}

%Ver Carro

\begin{enumerate}
	\item {\it Identificador:} Ver Carro.EscenarioNormal.CasoPrueba1
	\item {\it Situaciones iniciales:}
    		\begin{itemize}
    			\item No hay artículos en el carro
		\end{itemize}
	\item {\it Mensajes:}
		\begin{itemize}
			\item verCarro()
				 \begin{itemize}
					 	\item Resultado esperado:
					 \begin{itemize}
		       			 \item Mostrar el carro vacío.
					 \end{itemize}
				 \end{itemize}
		\end{itemize}
\end{enumerate}

\begin{enumerate}
	\item {\it Identificador:} Ver Carro.EscenarioNormal.CasoPrueba2
	\item {\it Situaciones iniciales:}
    		\begin{itemize}
    			\item Hay 3 productos en el carro. El producto 001, botas a 70 euros y hay 1 par de ellos. El producto 002, pantalones a 30 euros y hay 2. El producto 003, camisas a 20 euros y hay 1.
    			\item Hay 3 productos (4 artículos) en el carro
		\end{itemize}
	\item {\it Mensajes:}
		\begin{itemize}
			\item verCarro()
				 \begin{itemize}
					 	\item Resultado esperado:
					 \begin{itemize}
					 	 \item Hay 3 productos: 
		       			 \begin{itemize}
				   			\item Hay 1 par de botas, precio 70 euros.
				  			\item Hay 2 pantalones, precio 60 euros.
				  			\item Hay 1 camisa, precio 20 euros.
				   		\end{itemize}
				   		\item Precio total: 150 euros.
					 \end{itemize}
				 \end{itemize}
		\end{itemize}
\end{enumerate}


%Identificacion

\begin{enumerate}
	\item {\it Identificador:} Identificacion.EscenarioNormal.CasoPrueba1
	\item {\it Situaciones iniciales:}
    		\begin{itemize}
    			\item El visitante se logea correctamente como cliente.
		\end{itemize}
	\item {\it Mensajes:}
		\begin{itemize}
			\item login(nick,password)
				 \begin{itemize}
					 	\item Resultado esperado:
					 \begin{itemize}
					 	 \item Mostrar mensaje de confirmación.
					 \end{itemize}
				 \end{itemize}
		\end{itemize}
\end{enumerate}

\begin{enumerate}
	\item {\it Identificador:} Identificacion.EscenarioNormal.CasoPrueba2
	\item {\it Situaciones iniciales:}
    		\begin{itemize}
    			\item El visitante se logea correctamente como administrador.
		\end{itemize}
	\item {\it Mensajes:}
		\begin{itemize}
			\item login(nick,password)
				 \begin{itemize}
					 	\item Resultado esperado:
					 \begin{itemize}
					 	 \item Mostrar mensaje de confirmación.
					 \end{itemize}
				 \end{itemize}
		\end{itemize}
\end{enumerate}


\begin{enumerate}
	\item {\it Identificador:} Identificacion.EscenarioAlternativo.CasoPrueba3
	\item {\it Situaciones iniciales:}
    		\begin{itemize}
    			\item El visitante introduce un nombre de usuario inexistente.
		\end{itemize}
	\item {\it Mensajes:}
		\begin{itemize}
			\item login(nick,password)
				 \begin{itemize}
					 	\item Resultado esperado:
					 \begin{itemize}
					 	 \item Mostrar mensaje de error: Usuario o contraseña incorrectos.
					 \end{itemize}
				 \end{itemize}
		\end{itemize}
\end{enumerate}

\begin{enumerate}
	\item {\it Identificador:} Identificacion.EscenarioAlternativo.CasoPrueba4
	\item {\it Situaciones iniciales:}
    		\begin{itemize}
    			\item El visitante introduce una contraseña incorrecta.
		\end{itemize}
	\item {\it Mensajes:}
		\begin{itemize}
			\item login(nick,password)
				 \begin{itemize}
					 	\item Resultado esperado:
					 \begin{itemize}
					 	 \item Mostrar mensaje de error: El usuario o contraseña incorrectos.
					 \end{itemize}
				 \end{itemize}
		\end{itemize}
\end{enumerate}

%Registro

\begin{enumerate}
	\item {\it Identificador:} Registro.EscenarioNormal.CasoPrueba1
	\item {\it Situaciones iniciales:}
    		\begin{itemize}
    			\item El visitante introduce al menos los datos mínimos correctamente.
		\end{itemize}
	\item {\it Mensajes:}
		\begin{itemize}
			\item registrar(nick,password,email)
				 \begin{itemize}
					 	\item Resultado esperado:
					 \begin{itemize}
					 	 \item Mostrar mensaje de registro aceptado.
					 \end{itemize}
				 \end{itemize}
		\end{itemize}
\end{enumerate}

\begin{enumerate}
	\item {\it Identificador:} Registro.EscenarioAlternativo.CasoPrueba2
	\item {\it Situaciones iniciales:}
    		\begin{itemize}
    			\item El visitante introduce un nombre de usuario ya existente.
		\end{itemize}
	\item {\it Mensajes:}
		\begin{itemize}
			\item registrar(nick,password,email)
				 \begin{itemize}
					 	\item Resultado esperado:
					 \begin{itemize}
					 	 \item Mostrar mensaje error: Nombre de usuario ya existente.
					 \end{itemize}
				 \end{itemize}
		\end{itemize}
\end{enumerate}

\begin{enumerate}
	\item {\it Identificador:} Registro.EscenarioAlternativo.CasoPrueba2
	\item {\it Situaciones iniciales:}
    		\begin{itemize}
    			\item El visitante introduce un correo electrónico que ya esta asociado a una cuenta.
		\end{itemize}
	\item {\it Mensajes:}
		\begin{itemize}
			\item registrar(nick,password,email)
				 \begin{itemize}
					 	\item Resultado esperado:
					 \begin{itemize}
					 	 \item Mostrar mensaje error: Correo electrónico ya asociado a otra cuenta.
					 \end{itemize}
				 \end{itemize}
		\end{itemize}
\end{enumerate}


% Añadir Producto

\begin{enumerate}
\item {\it Identificador:} AñadirProducto.EscenarioNormal.CasoPrueba1
\item {\it Situaciones iniciales:}
  \begin{itemize}
  \item Hay 4 productos en la base de datos el producto 001 botas a 70 euros y hay 50 pares de unidades, el producto 002 pantalones a 30 euros y hay 20 pares, el producto 003 camisas a 20 euros y hay 200 y el producto 004 gorras a 10 euros y hay 10.
  \end{itemize}
\item {\it Mensajes:}
  \begin{itemize}
  \item añadirProducto(005,20,15,``Camisetas de platero y tú")
    \begin{itemize}
    \item {\it Resultado esperado:}
      \begin{itemize}
      \item Hay 5 productos.
        \begin{itemize}
        \item Hay 50 pares de botas a 70 euros.
        \item Hay 20 pares de pantalones a 50 euros.
        \item Hay 20 camisas a 30 euros.
        \item Hay 10 gorras a 10 euros.
        \item Hay 20 camisetas de platero y tú 15 euros.
        \end{itemize}
      \end{itemize}
    \end{itemize}
  \end{itemize}


\end{enumerate}

\begin{enumerate}
\item {\it Identificador:} AñadirProducto.EscenarioAlternativo.CasoPrueba2
\item {\it Situaciones iniciales:}
  \begin{itemize}
  \item Hay 4 productos en la base de datos el producto 001 botas a 70 euros y hay 50 pares de unidades, el producto 002 pantalones a 30 euros y hay 20 pares, el producto 003 camisas a 20 euros y hay 200 y el producto 004 gorras a 10 euros y hay 10.
  \end{itemize}
\item {\it Mensajes:}
  \begin{itemize}
  \item añadirProducto(003,20,15,``camisas")
    \begin{itemize}
    \item {\it Resultado esperado:}
      \begin{itemize}
      \item Hay 4 productos.
        \begin{itemize}
        \item Hay 50 pares de botas a 70 euros.
        \item Hay 20 pares de pantalones a 50 euros.
        \item Hay 20 camisas a 30 euros.
        \item Hay 10 gorras a 10 euros.
        \end{itemize}
      \end{itemize}
    \end{itemize}
  \end{itemize}


\end{enumerate}

\begin{enumerate}
\item {\it Identificador:} AñadirProducto.EscenarioAlternativo.CasoPrueba3
\item {\it Situaciones iniciales:}
  \begin{itemize}
  \item Hay 4 productos en la base de datos el producto 001 botas a 70 euros y hay 50 pares de unidades, el producto 002 pantalones a 30 euros y hay 20 pares, el producto 003 camisas a 20 euros y hay 200 y el producto 004 gorras a 10 euros y hay 10.
  \end{itemize}
\item {\it Mensajes:}
  \begin{itemize}
  \item añadirProducto(005,0,15,``camisas de platero y tú")
    \begin{itemize}
    \item {\it Resultado esperado:}
      \begin{itemize}
      \item Hay 4 productos.
        \begin{itemize}
        \item Hay 50 pares de botas a 70 euros.
        \item Hay 20 pares de pantalones a 50 euros.
        \item Hay 20 camisas a 30 euros.
        \item Hay 10 gorras a 10 euros.
        \end{itemize}
      \end{itemize}
    \end{itemize}
  \end{itemize}



\end{enumerate}

%Borrar Producto

\begin{enumerate}
\item {\it Identificador:} BorrarProducto.EscenarioNormal.CasoPrueba1
\item {\it Situaciones iniciales:}
  \begin{itemize}
  \item Hay 4 productos en la base de datos el producto 001 botas a 70 euros y hay 50 pares de unidades, el producto 002 pantalones a 30 euros y hay 20 pares, el producto 003 camisas a 20 euros y hay 200 y el producto 004 gorras a 10 euros y hay 10.
  \end{itemize}
\item {\it Mensajes:}
  \begin{itemize}
  \item BorrarProducto(004)
    \begin{itemize}
    \item {\it Resultado esperado:}
      \begin{itemize}
      \item Hay 3 productos.
        \begin{itemize}
        \item Hay 50 pares de botas a 70 euros.
        \item Hay 20 pares de pantalones a 50 euros.
        \item Hay 20 camisas a 30 euros.
        \end{itemize}
      \end{itemize}
    \end{itemize}
  \end{itemize}
\end{enumerate}

\begin{enumerate}
\item {\it Identificador:} BorrarProducto.EscenarioAlternativo.CasoPrueba2
\item {\it Situaciones iniciales:}
  \begin{itemize}
  \item Hay 4 productos en la base de datos el producto 001 botas a 70 euros y hay 50 pares de unidades, el producto 002 pantalones a 30 euros y hay 20 pares, el producto 003 camisas a 20 euros y hay 200 y el producto 004 gorras a 10 euros y hay 10.
  \end{itemize}
\item {\it Mensajes:}
  \begin{itemize}
  \item añadirProducto(005)
    \begin{itemize}
    \item {\it Resultado esperado:}
      \begin{itemize}
      \item Hay 4 productos.
        \begin{itemize}
        \item Hay 50 pares de botas a 70 euros.
        \item Hay 20 pares de pantalones a 50 euros.
        \item Hay 20 camisas a 30 euros.
        \item Hay 10 gorras a 10 euros.
        \end{itemize}
      \end{itemize}
    \end{itemize}
  \end{itemize}
\end{enumerate}

\section{Diagrama de clases}

El diagrama completo de clases puede apreciarse en el anexo \ref{clases}.


% \chapter{Fase de construcción}\label{CAPConstruccion}
% \input{doc/construccion.tex}

% \chapter{Fase de transición}\label{CAPTransicion}
% \input{doc/transicion.tex}

% \chapter{Conclusiones}\label{CAPConclusiones}
% \input{doc/conclusiones.tex}

% %%

% \chapter{Introducción}\label{CAPintro}
% \capital{V}estibulum sit amet arcu laoreet quam scelerisque lobortis in eget quam. Suspenkkdisse ornare tincidunt purus, eget imperdiet massa lacinia nec. Duis blandit lacus sit amet dolor auctor quis viverra libero fermentum. Aliquam id sapien velit, tempor ullamcorper dolor. Aliquam id sapien velit, tempor ullamcorper dolor.

Aenean eros lacus, blandit vel dictum sed, elementum at \index{metus}. Morbi sollicitudin urna eget sem condimentum non vehicula massa malesuada. Nullam laoreet aliquam interdum.

\section{Características de la plantilla}

Suspendisse aliquet diam pellentesque lectus posuere eu vehicula velit aliquet. Cras iaculis, quam quis venenatis venenatis, nulla leo porta mi, lacinia hendrerit arcu lorem et dui. Proin viverra pulvinar mi vel ullamcorper. Aliquam erat volutpat. Etiam ut velit nisl, in porta magna. Nullam varius dui sit amet dui scelerisque consectetur. Nullam tortor est, ullamcorper viverra ullamcorper a, convallis et nisl. Phasellus nunc ligula, fermentum quis iaculis at, consectetur vel diam. Aliquam tincidunt orci eget ante mollis pulvinar. In hac habitasse platea dictumst. Suspendisse id tellus a ante mattis feugiat. 

\subsection{Notas al margen}

\notamargen{Tiitulooo}{
Aliquam tincidunt orci eget ante mollis pulvinar. In hac habitasse platea dictumst. Nullam varius dui sit amet dui scelerisque consectetur. Nullam tortor est, ullamcorper viverra ullamcorper a, convallis et nisl.
}

Nullam varius dui sit amet dui scelerisque consectetur. Nullam tortor est, ullamcorper viverra ullamcorper a, convallis et nisl. Phasellus nunc ligula, fermentum quis iaculis at, consectetur vel diam. Aliquam tincidunt orci eget ante mollis pulvinar. In hac habitasse platea dictumst. Suspendisse id tellus a ante mattis feugiat. 

\subsection{Teclazos}

Phasellus nunc ligula, fermentum quis iaculis at, consectetur vel diam. Aliquam tincidunt orci eget ante mollis pulvinar. Nullam varius dui sit amet dui scelerisque consectetur. Nullam tortor est, ullamcorper viverra ullamcorper a, convallis et nisl. Phasellus nunc ligula, fermentum quis iaculis at, consectetur vel diam. In hac \teclazo{Alt}+\teclazo{F4} habitasse platea dictumst. Suspendisse id tellus a ante mattis feugiat. 

\imagenmargen{./img/feo.jpg}{labelx}{Nunc vitae porttitor dolor.}

\begin{center}
\teclazo{Ctrl}+\teclazo{Alt}+\teclazo{Supr}
\end{center}

Aliquam tincidunt orci eget ante mollis pulvinar. In hac habitasse platea dictumst. Nullam varius dui sit amet dui scelerisque consectetur. Nullam tortor est, ullamcorper viverra ullamcorper a, convallis et nisl.

\subsection{Imágenes al margen}

Aenean venenatis tortor vitae sapien tincidunt lobortis mollis enim molestie. Nam justo dui, pretium sed vulputate nec, convallis at ligula. Curabitur cursus volutpat dui, vitae auctor eros pulvinar in. 

Aenean urna enim, mollis ut tincidunt et, congue quis elit. Vestibulum ante ipsum primis in faucibus orci luctus et ultrices posuere cubilia Curae; Aliquam non ipsum ac felis lobortis blandit sed sit amet erat. Donec dictum, erat ac lobortis bibendum, velit lacus aliquam magna, at venenatis risus leo at mi. Phasellus lobortis, magna vestibulum suscipit pellentesque, lectus dolor rutrum ligula, a ornare nisi lectus quis orci. Nunc vitae porttitor dolor. 
\imagenmargen{./img/zoey.jpg}{labelxx}{Pellentesque habitant morbi tristique senectus et netus et malesuada fames ac turpis egestas. Integer dictum libero quis justo dapibus mattis.}

Pellentesque habitant morbi tristique senectus et netus et malesuada fames ac turpis egestas. Integer dictum libero quis justo dapibus mattis. Nullam lobortis ultricies ultricies. Suspendisse aliquam malesuada tincidunt. Morbi adipiscing orci ac lorem placerat aliquet. Morbi porta, lacus non eleifend hendrerit, tellus ipsum rhoncus arcu, sit amet ultricies velit sapien at neque. Vivamus sollicitudin ornare adipiscing. Sed imperdiet tellus volutpat nunc pulvinar luctus. Duis est urna, molestie eget ornare sit amet, fermentum et dui. 

\subsection{Bloques llamativos}

\imagenmargen{./img/feo.jpg}{labelxxx}{Nunc vitae porttitor dolor.}

Morbi porta, lacus non eleifend hendrerit, tellus ipsum rhoncus arcu, sit amet ultricies velit sapien at neque. Vivamus sollicitudin ornare adipiscing. Sed imperdiet tellus volutpat nunc pulvinar luctus. Duis est urna, molestie eget ornare sit amet, fermentum et dui. 

\importante{info}{Morbi porta, lacus non eleifend hendrerit, tellus ipsum rhoncus arcu, sit amet ultricies velit sapien at neque.}

Morbi porta, lacus non eleifend hendrerit, tellus ipsum rhoncus arcu, sit amet ultricies velit sapien at neque. Vivamus sollicitudin ornare adipiscing. Sed imperdiet tellus volutpat nunc pulvinar luctus. Duis est urna, molestie eget ornare sit amet, fermentum et dui. 

Pellentesque habitant morbi tristique senectus et netus et malesuada fames ac turpis egestas. Integer dictum libero quis justo dapibus mattis. Nullam lobortis ultricies ultricies. Suspendisse aliquam malesuada tincidunt. Morbi adipiscing orci ac lorem placerat aliquet. Morbi porta, lacus non eleifend hendrerit, tellus ipsum rhoncus arcu, sit amet ultricies velit sapien at neque. Vivamus sollicitudin ornare adipiscing. Sed imperdiet tellus volutpat nunc pulvinar luctus. Duis est urna, molestie eget ornare sit amet, fermentum et dui. 

\importante{warning}{Morbi porta, lacus non eleifend hendrerit, tellus ipsum rhoncus arcu, sit amet ultricies velit sapien at neque. Duis est urna, molestie eget ornare sit amet, fermentum et dui. }

Pellentesque habitant morbi tristique senectus et netus et malesuada fames ac turpis egestas. 

\imagenmargen{./img/mortadelo}{labelxxxx}{Nunc vitae porttitor dolor.}

Integer dictum libero quis justo dapibus mattis. Nullam lobortis ultricies ultricies. Suspendisse aliquam malesuada tincidunt. Morbi adipiscing orci ac lorem placerat aliquet. Morbi porta, lacus non eleifend hendrerit, tellus ipsum rhoncus arcu, sit amet ultricies velit sapien at neque. Vivamus sollicitudin ornare adipiscing. Sed imperdiet tellus volutpat nunc pulvinar luctus. Duis est urna, molestie eget ornare sit amet, fermentum et dui. 

\importante{question}{Morbi porta, lacus non eleifend hendrerit, tellus ipsum rhoncus arcu, sit amet ultricies velit sapien at neque. }

Morbi porta, lacus non eleifend hendrerit, tellus ipsum rhoncus arcu, sit amet ultricies velit sapien at neque. Vivamus sollicitudin ornare adipiscing. Sed imperdiet tellus volutpat nunc pulvinar luctus. Duis est urna, molestie eget ornare sit amet, fermentum et dui. 


\subsection{Imágenes aquí}

\index{Aenean} venenatis tortor vitae sapien tincidunt lobortis mollis enim molestie. Nam justo dui, pretium sed vulputate nec, convallis at ligula. Curabitur cursus volutpat dui, vitae auctor eros pulvinar in. 

Morbi porta, lacus non eleifend hendrerit, tellus ipsum rhoncus arcu, sit amet ultricies velit sapien at neque. Vivamus sollicitudin ornare adipiscing. Sed imperdiet tellus volutpat nunc pulvinar luctus. Duis est urna, molestie eget ornare sit amet, fermentum et dui. 

\imagenhere{./img/esi.png}{3cm}{Esto es el pie de foto}{labelxx0x0}

\imagenmargen{./img/manolobenito}{label002}{Nunc vitae porttitor dolor.}

Vivamus sollicitudin ornare adipiscing. Sed imperdiet tellus volutpat nunc pulvinar luctus.

\subsection{Grafos DOT empotrados (no son imágenes)}

Pellentesque posuere mattis pulvinar. Cras tempor aliquet \index{metus}. Etiam vitae pharetra nunc. Integer erat magna, tempor in tincidunt vel, lacinia a dui. Integer a nulla eu dolor laoreet vehicula. Sed non condimentum dui. Nulla rhoncus justo quis orci auctor quis rhoncus velit ultricies. Sed ac est ultrices ipsum gravida congue. 

\begin{figure}[h]
\begin{center}
  \begin{dot2tex}[dot,scale=1,mathmode]
    digraph G {
      rankdir=TB;
      d2ttikzedgelabels = false;
      node [style="state"];
      edge [lblstyle="auto"];
      X [label = "\pi"];
      A -> B [label = "\displaystyle\frac{x_2}{\sum y_i}"];
      A -> D [label = "7"];
      A -> X;
      X -> D;
      X->B;
      A->D [label = "\dfrac{1}{3}"];
    }
  \end{dot2tex}
\caption{Grafo empotrado}
\label{grafo03}
\end{center}
\end{figure}


Curabitur diam lacus, luctus et pretium non, gravida ut eros. Praesent aliquet lectus id mauris placerat ac rutrum risus gravida. Proin arcu magna, condimentum eget dictum ac, euismod convallis lorem. In varius erat accumsan quam luctus ac cursus libero vehicula. Curabitur ut sem ac eros tincidunt sollicitudin. Mauris eget eleifend lorem. Donec pulvinar neque id sem egestas ornare.

Curabitur diam lacus, luctus et pretium non, gravida ut eros. Praesent aliquet lectus id mauris placerat ac rutrum risus gravida. Proin arcu magna, condimentum eget dictum ac, euismod convallis lorem. 

In varius erat accumsan quam luctus ac cursus libero vehicula. Curabitur ut sem ac eros tincidunt sollicitudin. Mauris eget eleifend lorem. Donec pulvinar neque id sem egestas ornare.

\begin{figure}[h]
\begin{center}
  \begin{dot2tex}[dot,scale=1,mathmode]
    graph G {
      e
      subgraph clusterA {
        a -- b;
        subgraph clusterC {
          C -- D;
        }
      }
      subgraph clusterB {
        d -- f
      }
      d -- D
      e -- f
    } 
  \end{dot2tex}
\caption{Grafo empotrado}
\label{grafo04}
\end{center}
\end{figure}

Curabitur diam lacus, lctus et pretium non, gravida ut eros. Praesent aliquet lectus id mauris placerat ac rutrum risus gravida. Proin arcu magna, condimentum eget dictum ac, euismod convallis lorem. 

In varius erat accumsan quam luctus ac cursus libero vehicula. Curabitur ut sem ac eros tincidunt sollicitudin. Mauris eget eleifend lorem. Donec pulvinar neque id sem egestas ornare.

\begin{figure}[h]
\begin{center}
  \begin{dot2tex}[dot,scale=1,mathmode]
digraph otro {
ratio = "auto" ;
mincross = 2.0 ;
"001" [shape=box     , regular=1,style=filled,fillcolor=white   ] ;
"002" [shape=box     , regular=1,style=filled,fillcolor=white   ] ;
"003" [shape=circle  , regular=1,style=filled,fillcolor=white   ] ;
"004" [shape=box     , regular=1,style=filled,fillcolor=white   ] ;
"005" [shape=box     , regular=1,style=filled,fillcolor=white   ] ;
"006" [shape=circle  , regular=1,style=filled,fillcolor=white   ] ;
"007" [shape=circle  , regular=1,style=filled,fillcolor=white   ] ;

marr001 [shape=diamond,style=filled,label="",height=.1,width=.1] ;
marr002 [shape=diamond,style=filled,label="",height=.1,width=.1] ;

001 -> marr001
003 -> marr001
marr001 -> 007

004 -> marr002
005 -> marr002
marr002 -> 006

007 -> 002
006 -> 002

006 -> e

}
  \end{dot2tex}
\caption{Grafo empotrado}
\label{grafo05}
\end{center}
\end{figure}

In varius erat accumsan quam luctus ac cursus libero vehicula. Curabitur ut sem ac eros tincidunt sollicitudin. Mauris eget eleifend lorem. Donec pulvinar neque id sem egestas ornare.

\subsection{Terminal empotrado}

Curabitur diam lacus, lctus et pretium non, gravida ut eros. Praesent aliquet lectus id mauris placerat ac rutrum risus gravida. Proin arcu magna, condimentum eget dictum ac, euismod convallis lorem. 

In varius erat accumsan quam luctus ac cursus libero vehicula. Curabitur ut sem ac eros tincidunt sollicitudin. Mauris eget eleifend lorem. Donec pulvinar neque id sem egestas ornare.

\begin{term}
# apt-get update
# apt-get upgrade
$ sudo apt-get install emacs
$ sudo pacman -S emacs auctex
\end{term}

In varius erat accumsan quam luctus ac cursus libero vehicula. Curabitur ut sem ac eros tincidunt sollicitudin. Mauris eget eleifend lorem. Donec pulvinar neque id sem egestas ornare. Curabitur diam lacus, lctus et pretium non, gravida ut eros. Praesent aliquet lectus id mauris placerat ac rutrum risus gravida. Proin arcu magna, condimentum eget dictum ac, euismod convallis lorem. 

\subsection{Código fuente empotrado}

In varius erat accumsan quam luctus ac cursus libero vehicula. Curabitur ut sem ac eros tincidunt sollicitudin. Mauris eget eleifend lorem. Donec pulvinar neque id sem egestas ornare. Curabitur diam lacus, lctus et pretium non, gravida ut eros. Praesent aliquet lectus id mauris placerat ac rutrum risus gravida. Proin arcu magna, condimentum eget dictum ac, euismod convallis lorem. In varius erat accumsan quam luctus ac cursus libero vehicula. Curabitur ut sem ac eros tincidunt sollicitudin. Mauris eget eleifend lorem. Donec pulvinar neque id sem egestas ornare. Curabitur diam lacus, lctus et pretium non, gravida ut eros. Praesent aliquet lectus id mauris placerat ac rutrum risus gravida. Proin arcu magna, condimentum eget dictum ac, euismod convallis lorem. 

In varius erat accumsan quam luctus ac cursus libero vehicula. Curabitur ut sem ac eros tincidunt sollicitudin. Mauris eget eleifend lorem. Donec pulvinar neque id sem egestas ornare. Curabitur diam lacus, lctus et pretium non, gravida ut eros. 

\code{C}{900}{Hola mundo en GO}{./src/hello.go}

\imagenmargen{./img/esi.png}{labelvvvv}{Esto es el logo de la ESI con un pie de imagen. Este pie de imagen es más largo, con el fin de comprobar que la alineación del mismo es justificada.}

Curabitur diam lacus, lctus et pretium non, gravida ut eros. Praesent aliquet lectus id mauris placerat ac rutrum risus gravida. Proin arcu magna, condimentum eget dictum ac, euismod convallis lorem. 

\code{C}{6}{Hola mundo en C}{./src/hello.c}

Curabitur diam lacus, lctus et pretium non, gravida ut eros. Praesent aliquet lectus id mauris placerat ac rutrum risus gravida. Proin arcu magna, condimentum eget dictum ac, euismod convallis lorem. 

\subsection{Imágenes de ancho total}

Curabitur diam lacus, lctus et pretium non, gravida ut eros. Praesent aliquet lectus id mauris placerat ac rutrum risus gravida. Proin arcu magna, condimentum eget dictum ac, euismod convallis lorem. In varius erat accumsan quam luctus ac cursus libero vehicula. Curabitur ut sem ac eros tincidunt sollicitudin. 

\imagenanchototal{./img/ancha.jpg}{Pie de foto ancha}{etiquetadefotoancha}

In varius erat accumsan quam luctus ac cursus libero vehicula. Curabitur ut sem ac eros tincidunt sollicitudin. Mauris eget eleifend lorem. Donec pulvinar neque id sem egestas ornare. Curabitur diam lacus, lctus et pretium non, gravida ut eros. Praesent aliquet lectus id mauris placerat ac rutrum risus gravida. Proin arcu magna, condimentum eget dictum ac, euismod convallis lorem. In varius erat accumsan quam luctus ac cursus libero vehicula. Curabitur ut sem ac eros tincidunt sollicitudin. Mauris eget eleifend lorem. Donec pulvinar neque id sem egestas ornare. Curabitur diam lacus, lctus et pretium non, gravida ut eros. Praesent aliquet lectus id mauris placerat ac rutrum risus gravida. Proin arcu magna, condimentum eget dictum ac, euismod convallis lorem. 

\imagenanchototal{./img/coche.jpg}{Pie de foto ancha segunda}{etiquetadefotoancha2}

Mauris eget eleifend lorem. Donec pulvinar neque id sem egestas ornare. Curabitur diam lacus, lctus et pretium non, gravida ut eros. Praesent aliquet lectus id mauris placerat ac rutrum risus gravida. Proin arcu magna, condimentum eget dictum ac, euismod convallis lorem. 

esto es un ejemplo para \index{términos} del glosario

In varius erat accumsan quam luctus ac cursus libero vehicula. Curabitur ut sem ac eros tincidunt sollicitudin. Mauris eget eleifend lorem. Donec pulvinar neque id sem egestas ornare. Curabitur diam lacus, lctus et pretium non, gravida ut eros. Praesent aliquet lectus id mauris placerat ac rutrum risus gravida. Proin arcu magna, condimentum eget dictum ac, euismod convallis lorem. In varius erat accumsan quam luctus ac cursus libero vehicula. Curabitur ut sem ac eros tincidunt sollicitudin. Mauris eget eleifend lorem. Donec pulvinar neque id sem egestas ornare. Curabitur diam lacus, lctus et pretium non, gravida ut eros. Praesent aliquet lectus id mauris placerat ac rutrum risus gravida. Proin arcu magna, condimentum eget dictum ac, euismod convallis lorem. 

\section{Lo de siempre}

In varius erat accumsan quam luctus ac cursus libero vehicula. Curabitur ut sem ac eros tincidunt sollicitudin. Mauris eget eleifend lorem. 

Donec pulvinar neque id sem egestas ornare. Curabitur diam lacus, lctus et pretium non, gravida ut eros. Praesent aliquet lectus id mauris placerat ac rutrum risus gravida. Proin arcu magna, condimentum eget dictum ac, euismod convallis lorem. In varius erat accumsan quam luctus ac cursus libero vehicula. Curabitur ut sem ac eros tincidunt sollicitudin. Mauris eget eleifend lorem. Donec pulvinar neque id sem egestas ornare. Curabitur diam lacus, lctus et pretium non, gravida ut eros. Praesent aliquet lectus id mauris placerat ac rutrum risus gravida. Proin arcu magna, condimentum eget dictum ac, euismod convallis lorem. 




% \chapter{Conjuntos Difusos y Variables Lingüísticas}\label{CAPFuzzySets}
% Esto es \index{conjuntos}. Esto es conjuntos. Esto es conjuntos. Esto es conjuntos. Esto es conjuntos. Esto es conjuntos. Esto es conjuntos. Esto es conjuntos. Esto es conjuntos. Esto es conjuntos. 

Esto es conjuntos. Esto es conjuntos. Esto es conjuntos. Esto es conjuntos. Esto es conjuntos. Esto es conjuntos. 

Esto es conjuntos. Esto es conjuntos. Esto es conjuntos. Esto es conjuntos. Esto es conjuntos. Esto es conjuntos. 

Esto es conjuntos. Esto es conjuntos. Esto es conjuntos. Esto es conjuntos. Esto es conjuntos. Esto es conjuntos. Esto es conjuntos. Esto es conjuntos. Esto es conjuntos. Esto es conjuntos. Esto es conjuntos. Esto es conjuntos. Esto es conjuntos. Esto es conjuntos. 

% \chapter{Razonamiento Aproximado}\label{CAPRazAprox}
% Esto es reglas. Esto es reglas. Esto es reglas. Esto es reglas. Esto es reglas. Esto es reglas. Esto es reglas. 

Esto es reglas. Esto es reglas. Esto es reglas. Esto es reglas. Esto es reglas. Esto es reglas. 

Esto es reglas. Esto es reglas. Esto es reglas. Esto es reglas. Esto es reglas. Esto es reglas. Esto es reglas. Esto es reglas. Esto es reglas. Esto es reglas. Esto es reglas. Esto es reglas. Esto es reglas. 

% \chapter{Conclusiones}\label{CAPConclusiones}
% Esto es reglas. Esto es reglas. Esto es reglas. Esto es reglas. Esto es reglas. Esto es reglas. Esto es reglas. 

Esto es reglas. Esto es reglas. Esto es reglas. Esto es reglas. Esto es reglas. Esto es reglas. 

Esto es reglas. Esto es reglas. Esto es reglas. Esto es reglas. Esto es reglas. Esto es reglas. Esto es reglas. Esto es reglas. Esto es reglas. Esto es reglas. Esto es reglas. Esto es reglas. Esto es reglas. 

% \appendix
% \chapter{Sed purus quam}\label{APSedPurusQuam}
% \input{doc/apendiceA.tex}

% \chapter{Malius mostimer}\label{APMaliusMostimer}
% \capital{U}tte facilisis dapibus lacus eu ullamcorper. Sed gravida urna at nunc aliquam congue. Nullam et ante tellus, quis adipiscing purus. Phasellus quis fermentum mi. Pellentesque malesuada convallis nunc nec euismod. Morbi adipiscing lectus ullamcorper urna aliquam ut euismod arcu sagittis. Etiam laoreet ultricies libero et tempus. Mauris porttitor, nulla eu iaculis eleifend, dui augue condimentum nisi, non ullamcorper nisi sem ut lacus. Duis ac nulla nisl, sed volutpat mauris. Lorem ipsum dolor sit amet, consectetur adipiscing elit. Aliquam erat volutpat. Nunc nunc nulla, placerat faucibus viverra non, iaculis vel nunc. Curabitur quis dui ligula, venenatis porttitor augue. Proin libero mauris, volutpat sed vulputate non, dignissim nec nisl. In hac habitasse platea dictumst. Fusce turpis augue, rhoncus ac dapibus vitae, luctus sed felis. Nunc convallis iaculis bibendum. Vestibulum vel leo risus, nec dapibus felis. Proin commodo luctus tortor, ut congue est rutrum ac. Pellentesque habitant morbi tristique senectus et netus et malesuada fames ac turpis egestas.

Sed malesuada, sapien vitae placerat consequat, risus lectus consectetur erat, sit amet feugiat ante est sed arcu. Maecenas metus metus, ornare sit amet mollis ut, condimentum eget leo. Suspendisse blandit augue eu erat malesuada nec congue lectus posuere. Morbi imperdiet nisi eget orci laoreet tincidunt. Cras accumsan nisl quis quam sagittis. 

\section{Sed malesuada}

Nullam sagittis fringilla auctor. Quisque eu fringilla neque. Aliquam aliquam iaculis turpis, et pellentesque metus consectetur ac. Suspendisse potenti. Phasellus sem ligula, congue sit amet rutrum vel, faucibus id libero. Curabitur convallis imperdiet mi, quis aliquam arcu pharetra in. Morbi congue fermentum tortor. Integer gravida venenatis vulputate. Morbi erat sem, pellentesque id egestas ac, consequat eget neque. Nunc aliquet erat at magna commodo sit amet condimentum neque iaculis. Maecenas nec neque vitae purus iaculis scelerisque at eu leo. Donec non facilisis lorem. Aenean sed tellus lacus. Duis a est quis felis scelerisque sagittis nec nec orci.

Mauris vulputate nisi ut ante mollis fringilla. Proin in felis lectus. Lorem ipsum dolor sit amet, consectetur adipiscing elit. Sed at nisi non massa faucibus mattis ornare non quam. Aenean posuere mauris faucibus odio faucibus faucibus. Aenean ac est in mauris pulvinar hendrerit. Sed non tortor et ipsum dictum tincidunt. Donec vel ipsum mi, quis rhoncus leo. Proin dui erat, vehicula sed vehicula quis, laoreet sit amet urna. Donec congue felis sed libero dapibus a pretium arcu vulputate. 

\subsection{Donec elit velit}

Donec elit velit, porta id tristique vel, euismod eu massa. Donec egestas luctus dapibus. Aenean neque tellus, pellentesque eget rutrum mattis, vestibulum quis ligula. Vestibulum id est a massa egestas mattis ac in turpis. Nulla lacus nulla, facilisis ac placerat nec, consectetur in nibh. Sed vehicula est id nisi aliquam a gravida velit imperdiet. Quisque tortor augue, laoreet nec eleifend et, pulvinar non lorem. Etiam ac purus ac justo congue ultrices. Nunc commodo erat eget dolor tristique vel suscipit nulla commodo. Donec eu est ac tellus tincidunt porttitor eu eu dolor. Curabitur tristique gravida ipsum, at lobortis ligula suscipit quis. Nam dictum vulputate nulla ac volutpat. Vestibulum a quam purus. Sed neque purus, interdum quis tempus nec, pharetra sed ligula. Proin nec dui in lorem hendrerit sodales a sit amet nibh. Nam sem orci, feugiat ac dignissim non, accumsan vitae ante. 

Nullam ullamcorper adipiscing ultrices. Proin quis felis at risus ornare molestie a nec erat. Proin bibendum purus a odio dapibus ultrices. Maecenas pellentesque aliquet ipsum, vitae fermentum tortor ultricies eu. Suspendisse faucibus vehicula mi in dignissim. Phasellus non est a odio ultrices semper.

\subsection{Nullam ullamcorper}

Nullam ullamcorper adipiscing ultrices. Proin quis felis at risus ornare molestie a nec erat. Proin bibendum purus a odio dapibus ultrices. Maecenas pellentesque aliquet ipsum, vitae fermentum tortor ultricies eu. Suspendisse faucibus vehicula mi in dignissim. Phasellus non est a odio ultrices semper.


\section{Mauris vulputate}

Nunc iaculis semper augue et dictum. Suspendisse nunc mi, consectetur at auctor convallis, pulvinar eu odio. Quisque et orci vel tellus facilisis cursus. 

Sed id libero augue, vel condimentum ipsum. Duis enim massa, malesuada eget accumsan ac, euismod id orci. Aliquam vitae ante in sapien tempus euismod. Donec dolor felis, convallis ac dapibus a, lobortis eleifend est. 

Nunc a consequat libero. Curabitur ac pretium tortor. Suspendisse viverra enim eu ligula fermentum sit amet facilisis massa tincidunt. Etiam dictum luctus risus, a ultricies sapien suscipit id. Phasellus eleifend, nunc id mollis ullamcorper, nunc leo interdum risus, vitae egestas arcu tortor a sem. Nullam ullamcorper adipiscing ultrices. Proin quis felis at risus ornare molestie a nec erat. Proin bibendum purus a odio dapibus ultrices. Maecenas pellentesque aliquet ipsum, vitae fermentum tortor ultricies eu. Suspendisse faucibus vehicula mi in dignissim. Phasellus non est a odio ultrices semper.

Curabitur tristique gravida ipsum, at lobortis ligula suscipit quis. Nam dictum vulputate nulla ac volutpat.



% \nocite{*}
% \bibliographystyle{plain}
% \bibliography{bibliografia}

% \cleardoublepage
% \printindex


\newpage
%\setcounter{page}{1}
\setcounter{chapter}{0}
\def\thechapter{\Alph{chapter}}
\renewcommand{\chaptername}{Anexo }
\chapter{Lista de tareas}
\label{tareas}
\includepdfmerge{
doc/anexos/listaTareas.pdf,-
}
\chapter{Diagrama de casos de uso}
\label{casosuso}
%    \imagenanchototalimpar{diagramas/casosuso.png}{Diagrama de casos de uso}{casosuso}    
\chapter{Diagrama de clases}
\label{clases}
\url{./diagramas/diagrama_clases.png}
%\imagenanchototalimpar{diagramas/clases.png}{Diagrama de clases}{clases}    

\chapter{Diagrama de base de datos}

%\input{anexos/punteros.tex}
%\chapter{Dónde Seguir...}
%\input{anexos/dondeseguir.tex}

\end{document}
