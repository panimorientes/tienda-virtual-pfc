\section{Plan de proyecto}

Para nuestro proyecto tendremos reuniones de seguimiento al final de cada semana para ver que se han realizado las tareas asignadas a cada uno. Cada dos semanas quedaremos para debatir dudas sobre el proyecto y asignar nuevas tareas.

Una vez tengamos una cantidad considerable de información, David Antonio Pérez Zaba revisará el proyecto, y lo corregirá si fuese necesario.

\subsection{Estructura organizativa}

Los roles que se van a seguir para la realización de la práctica, a fin de que todos los integrantes del grupo participen en al menos un flujo de trabajo, son los siguientes:

\begin{enumerate}

\item \textbf{Fase de Inicio:}
  \begin{itemize}
  \item Requisitos:
    \begin{itemize}
    \item Sergio García Mondaray.
    \end{itemize}

  \item Análisis:
    \begin{itemize}
    \item Gabriel Alises Cano.
    \end{itemize}
    
  \item Diseño:
    \begin{itemize}
    \item David Antonio Pérez Zaba.
    \end{itemize}

 \item Implementación:
    \begin{itemize}
    \item Jorge Merino García.
    \end{itemize}
  \end{itemize}

\item \textbf{Fase de Elaboración:}
  \begin{itemize}
  \item Requisitos:
    \begin{itemize}
    \item Gabriel Alises Cano.
    \end{itemize}

  \item Análisis:
    \begin{itemize}
    \item David Antonio Pérez Zaba.
    \end{itemize}
    
  \item Diseño:
    \begin{itemize}
    \item Jorge Merino García.
    \end{itemize}

  \item Implementación:
    \begin{itemize}
    \item Sergio García Mondaray.
    \end{itemize}

  \item Pruebas:
    \begin{itemize}
    \item Gabriel Alises Cano.
    \end{itemize}
  \end{itemize}

\item \textbf{Fase de Construcción:}
  \begin{itemize}
  \item Requisitos:
    \begin{itemize}
    \item David Antonio Pérez Zaba.
    \end{itemize}

  \item Análisis:
    \begin{itemize}
    \item Jorge Merino García.
    \end{itemize}
    
  \item Diseño:
    \begin{itemize}
    \item Sergio García Mondaray.
    \end{itemize}

  \item Implementación:
    \begin{itemize}
    \item Gabriel Alises Cano.
    \end{itemize}

  \item Pruebas:
    \begin{itemize}
    \item David Antonio Pérez Zaba.
    \end{itemize}
  \end{itemize}

\item \textbf{Fase de Transición:}
  \begin{itemize}

  \item Análisis:
    \begin{itemize}
    \item Jorge Merino García.
    \end{itemize}
    
  \item Diseño:
    \begin{itemize}
    \item David Antonio Pérez Zaba
    \end{itemize}

  \item Implementación:
    \begin{itemize}
    \item Gabriel Alises Cano.
    \end{itemize}

  \item Pruebas:
    \begin{itemize}
    \item Jorge Merino García.
    \item Sergio García Mondaray.
    \end{itemize}

  \end{itemize}
\end{enumerate}


\subsection{Responsabilidades}
A nivel de asignatura los profesores se encargarán de proporcionar un sistema de pago externo a nuestra tienda, de manera que se asigna al profesor la responsabilidad de implementación de esta terea.
%\paragraph*{}

Dentro de nuestro grupo al tener repartidos los flujos de trabajo entre cada uno de nuestros integrantes todos tendremos que tener las responsabilidades acordes al estudio de los requisitos, análisis, diseño, implementación y pruebas en distintos momentos del desarrollo de la aplicación. Siendo siempre el jefe de proyecto David Antonio Pérez Zaba.

%\paragraph*{}
El grupo tendrá la responsabilidad de entregar una aplicación funcional el 15 de Noviembre de 2011 a más tardar, con una entrega parcial para el 14 de Octubre de 2011.

\subsection{Gestión de Riesgos}
Los principales riesgos a los que nos enfrentamos a la hora de realizar la aplicación son los siguientes:
\begin{itemize}
%\item No conseguir terminar la aplicación antes de la fecha de entrega.
\item Errores y tiempo de aprendizaje superior al estimado a la hora de utilizar una tecnología que desconocemos como Apache Tomcat.

\begin{itemize}
\item Probabilidad: Alta.

\item Efecto: Tolerable. Si nos retrasamos por no saber configurar estas herramientas, tendremos menos tiempo para elaborar la práctica.

\end{itemize}

\item Problemas y conflictos al utilizar varias personas el control de versiones que hemos elegido (Mercurial), como por ejemplo borrar el progreso de algún compañero o tu propio progreso.

\begin{itemize}
\item Probabilidad: Alta.

\item Efecto: Tolerable, si el sistema tiene el control de versiones actualizadas.

\end{itemize}


\item Dificultades para aprender correctamente HTML,  el lenguaje JavaScript, CSS, y cómo integrar Java con HTML.

\begin{itemize}
\item Probabilidad: Alta.

\item Efecto: Serio. Habrá que dedicar demasiado tiempo al principio para aprender, y puede que no se utilice correctamente debido al poco tiempo que hay para practicar ante de entregar la práctica.

\end{itemize}


\item El tiempo que se ha estimado para desrrollar la práctica se ha subestimado.

\begin{itemize}
\item Probabilidad: Alta.

\item Efecto: Catastrófico, lo que llevará a una sobrecarga de horas de trabajo y posible pérdida de calidad de las entregas.

\end{itemize}


\item Dificultades a la hora de planficar las reuniones, o falta de tiempo de algún miembro por los distintos itinerarios de asginaturas que poseen los miembros del grupo.
\item Errores por no conseguir enfocar el trabajo acorde con la metodología propuesta, es decir, el proceso unificado de desarrollo.

\begin{itemize}
\item Probabilidad: Media.

\item Efecto: Catastrófico. Posibles cambios en el análisis harán que haya que modificar el diseño, y sucesivas modificaciones en la implementación y pruebas.

\end{itemize}


\item Algún miembro del equipo abandona el grupo, o se cambia a otro.

\begin{itemize}
\item Probabilidad: Muy baja.

\item Efecto: Catastrófico. Habría que reorganizar el equipo buscando algún otro integrante o se podrucirá un retraso muy grande en la entrega de la práctica.

\end{itemize}


\item Algún miembro del equipo se encenutra temporalmente indispuesto por alguna enfermedad, o problema personal.

\begin{itemize}
\item Probabilidad: Media.

\item Efecto: Catastrófico. Habrá un sobrecarga de trabajo para los demás miembros del equipo.

\end{itemize}


\item Se da algún error o se han olvidado datos en el análisis y hay que volver a hacer el diseño.

\begin{itemize}
\item Probabilidad: Media.

\item Efecto: Tolerable, siempre y cuando los cambios no sean demasiado grandes, y se pueda solucionar en la siguiente interación.

\end{itemize}


\item Poca coordinación entre los miembros del equipo, por falta de trabajo del jefe de proyecto.

\begin{itemize}
\item Probabilidad: Baja.

\item Efecto: Tolerable. Se habrá creado trabajo repetido, o con poca cohesión.

\end{itemize}


\end{itemize}

\subsection{Métodos,  Herramientas,  Técnicas}
Para el desarrollo de la aplicación utilizaremos el proceso unificado de desarrollo, es decir, seguiremos un proceso iterativo e incremental para corrección de posibles errores en distintas fases.

%\paragraph*{}
Para el control de versiones utilizaremos mercurial para tener de una manera más organizada y poder realizar cambios en los archivos y tener toda la aplicación centralizada en un repositorio que podamos clonar desde cualquier computador con conexión a internet. La técnica que utilizaremos para actualizar el repositorio es ejecutar la instrucción pull antes de comenzar a utilizar cualquier archivo y así trabajar con la última versión de nuestra aplicación. Cuando terminemos nuestra sesión de trabajo notificaremos los cambios con la instrucción ci y posteriormente los subiremos al repositorio con la instrucción push.

%\paragraph*{}
Para implementar la funcionalidad de la aplicación utilizaremos diferentes tecnologías como las que siguen:
\begin{itemize}
\item Para el despliegue de la aplicación utilizaremos Tomcat6.0
\item Para la implementación de la aplicación utilizaremos JavaScript, Java, HTML y, Eclipse como entorno de desarrollo.
\item Para el diseño y análisis de requisitos utilizaremos Visual Paradigm, y Quick Sequence Diagram Editor.
\item Para la planificación de proyecto utilizaremos  Microsoft Project.
\item Para la generación de la documentación utilizaremos Latex.
\item Para la maquetación web utilizaremos plantillas css.
\item Para la coordinación del grupo, utilizaremos el servidor de \url{bitbucket.org} (wiki,rss) y el correo electrónico.
\end{itemize}

\notamargen{Diagrama de Gantt}{El diagrama de Gantt asociado al listado de tareas será mostrado en posteriores entregas.}

\subsection{Lista de tareas}

El conjunto de las tareas a desarrollar y su duración aproximada, así como las dependencias entre las tareas, hitos,... se encuentran en el anexo \ref{tareas}.

% diagrama de Gantt.
% \begin{center}
% \begin{figure}[H]
% \includegraphics[scale = 0.6 ]{./gantt.png}
% \end{figure}
% \end{center}
