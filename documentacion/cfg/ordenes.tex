\usepackage[utf8]{inputenc}
\usepackage{ifthen}
\usepackage{colortbl}
\usepackage{fancyvrb}
\usepackage{captdef}
\usepackage{fancybox}
\usepackage{lettrine}
\usepackage{shadow}
\usepackage{mparhack}
\usepackage{tabularx}
\usepackage{booktabs}

\definecolor{gris}{rgb}{0.3,0.3,0.3}
\definecolor{grisclaro}{rgb}{0.85,0.85,0.85}
\definecolor{grisoscuro}{rgb}{0.2,0.2,0.2}
\definecolor{comentarios}{rgb}{0.15, 0.15, 0.4}
\definecolor{grisconsola}{rgb}{0.85,0.85,0.85}

\renewcommand{\LettrineFontHook}{\color[gray]{0.4}}

\graphicspath{{images/}}
\reversemarginpar

\usepackage{fancyhdr}

%\setlenght\headheight{17pt}
\fancyhf{}
\fancyheadoffset[RO,LE]{0pt}
\fancyheadoffset[LO]{145pt}
\fancyheadoffset[RE]{145pt}
\fancyhead[RO,LE]{[\textbf{\thepage}]}
\fancyhead[LO]{\nouppercase{\rightmark}} % Seccion
\fancyhead[RE]{\leftmark}  % CAPITULO
\renewcommand\headrule{\hrule height 1.5pt width\headwidth \vspace{1mm}}

\usepackage{listings}
\usepackage{courier}
\renewcommand{\lstlistingname}{Listado}

% Para el dibujado de grafos (requiere tener instalado el paquete dot2tex en el sistema)
\usepackage[outputdir={./dot/},autosize]{dot2texi}
\usepackage{tikz}
\usetikzlibrary{shapes,arrows,automata}

% \lstset{
%          basicstyle=\footnotesize\ttfamily, 
%          numbers=left,               
%          numberstyle=\tiny,          
%          numbersep=10pt,              
%          tabsize=2,                  
%          extendedchars=true,         
%          breaklines=true,            
%          keywordstyle=\color{red},
%          frame=b,
%          framerule=1pt,
%          keywordstyle=\color{black}\bfseries,
%          stringstyle=\color{gris}\ttfamily, 
%          showspaces=false,           
%          showtabs=false,             
%          xleftmargin=17pt,
%          framexleftmargin=17pt,
%          framexrightmargin=5pt,
%          framexbottommargin=4pt,
%          commentstyle=\color{comentarios},
%          backgroundcolor=\color{grisclaro},
%          showstringspaces=false, 
%          escapeinside={(*@}{@*)}
%  }

\lstset{tabsize=4,
        showspaces=false,
        showtabs=false,
        frame=b,
        framerule=1pt,
        aboveskip=0.5cm,
        framextopmargin=3pt,
        framexbottommargin=3pt,
        framexleftmargin=18pt,
        framesep=.4pt,
        rulesep=.4pt,
        rulesepcolor=\color{grisoscuro},
        stringstyle=\ttfamily,
        showstringspaces = false,
        basicstyle=\footnotesize\ttfamily,
        commentstyle=\color{gris},
        keywordstyle=\bfseries,
        numbers=left,
        numbersep=6pt,
        numberstyle=\color[cmyk]{0.43, 0.35, 0.35,0.01}\bfseries\scriptsize\ttfamily,
        numberfirstline = true,        
        breaklines=true,
        stepnumber=1,
        backgroundcolor=\color{white},
        xleftmargin=18pt,
        framexrightmargin=0pt,
        xrightmargin=0pt
}

\usepackage{caption}
\DeclareCaptionFont{white}{\fontsize{9}{9}\selectfont\color{white}}
\DeclareCaptionFormat{listing}{\colorbox[cmyk]{0.43, 0.35, 0.35,0.01}{\parbox{0.982\textwidth}{\hspace{15pt}#1#2#3}}}
\captionsetup[lstlisting]{format=listing,labelfont=white,textfont=white, singlelinecheck=false, margin=0pt, font={bf,footnotesize}}

%%%%%%%%%%%%%%%%%%%%%%%%%%%%%%%%%%%%%%%%%%%%%%%%%%%%%%%%%%%%%%%%%%%%%%%%%%%%%%%%

\newcommand{\bigrule}{\vspace{-0.5cm}\hspace{-4.95cm}\titlerule[1.0mm]}

%%%%%%%%%%%%%%%%%%%%%%%%%%%%%%%%%%%%%%%%%%%%%%%%%%%%%%%%%%%%%%%%%%%%%%%%%%%%%%%%

\lstnewenvironment{term}{
  \interlineado{0.92}
  \lstset{
    basicstyle=\footnotesize\bf\ttfamily,
    numbers=none,
    xleftmargin=3.5pt,
    frame=none, 
    breaklines=true, 
    framexleftmargin=3pt,
    framexrightmargin=0pt,
    backgroundcolor=\color{grisclaro},
  }
}{\interlineado{1.0}}

%%%%%%%%%%%%%%% %%%%%%%%%%%%%%%%%%%%%%%%%%%%%%%%%%%%%%%%%%%%%%%%%%%%%%%%%%%%%%%%%


%%%%%%%%%%%%%%%%%%%%%%%%%%%%%%%%%%%%%%%%%%%%%%%%%%%%%%%%%%%%%%%%%%%%%%%%%%%%%%%%

 \lstloadlanguages{% Check Dokumentation for further languages ...
         C, Python
 }

\lstdefinestyle{C}
{language=C, }

\lstdefinestyle{P}
{language=Python, }

\lstdefinestyle{consola}
{
  basicstyle=\footnotesize\bf\ttfamily,
  backgroundcolor=\color{grisconsola},
}

\def\codigofuente#1#2#3#4{
  \interlineado{#4}
  \lstinputlisting[label=#3,caption=#2, style=C]{#1}
  \interlineado{1.0}
}

\lstnewenvironment{terminal}
    {\interlineado{0.92}\lstset{style=consola, numbers=none, frame=none, breaklines=true, frame=l, framexleftmargin=7pt}}
    {\interlineado{1.0}}



\def\comando#1{\texttt{\small{#1}}}

\def\linea#1{\fboxsep=0.8mm\fontsize{7}{7}\selectfont\ovalbox{\texttt{\textbf{#1}}}\normalsize}

\def\botonazo#1{\fboxsep=0.8mm\fontsize{6}{6}\selectfont\ovalbox{\textbf{#1}}\normalsize}


\def\raton#1{
\negthinspace \negthinspace
\begin{minipage}[c][1ex][c]{1em}
   \resizebox{!}{1em}{\includegraphics{botonazos/#1}}
\end{minipage}
}


\def\interfaz#1{
\negthinspace \negthinspace
\begin{minipage}[c][1ex][c]{1em}
   \resizebox{!}{1em}{\linethickness{0.3mm} \frame{\includegraphics{botonazos/#1}}}
\end{minipage}
}















 



%%%%%%%%%%%%%%%%%%%%%%%%%%%%%%%%%%%%%%%%%%%%%%%%%%%%%%%%%%%%%%%%%%%%%%



%% Insercion de teclazos %%%%%%%%%%%%%%%%%%%%%%%%%%%%%%%%%%%%%%%%%%%%%%%%%%%
%% Ejemplo: \teclazo{Ctrl} %%%%%%%%%%%%%%%%%%%%%%%%%%%%%%%%%%%%%%%%%%%%%%%%%

\def\teclazo#1{
    \negthinspace 
    \fontsize{9}{6}\selectfont \ovalbox{\texttt{\textbf{#1}}} \normalsize
    \negthinspace 
}

%% Imagen en el margen %%%%%%%%%%%%%%%%%%%%%%%%%%%%%%%%%%%%%%%%%%%%%%%%%%%%%
%% Ejemplo: \imagenmargen{path}{label}{caption} %%%%%%%%%%%%%%%%%%%%%%%%%%%%

\def\imagenmargen#1#2#3{
    \marginparsep=1cm
    \captionsetup{margin=0pt,font=footnotesize,labelfont=bf}
    \marginparwidth=4cm
    \marginpar{ 
        \resizebox{\marginparwidth}{!}{\includegraphics{#1}}
    }
    \marginpar{
        \vspace{0.2cm}
        \figcaption{#3}\label{#2}
    }
}

%% Nota en el margen %%%%%%%%%%%%%%%%%%%%%%%%%%%%%%%%%%%%%%%%%%%%%%%%%
%% Ejemplo: \notamargen{Texto} %%%%%%%%%%%%%%%%%%%%%%%%%%%%%%%%%%%%%%%

\def\notamargen#1#2{
    \interlineado{1.3}
    \ifodd\thepage
        \marginparsep=1.0cm
    \else
        \marginparsep=1.0cm
    \fi 
    \marginparwidth=4.0cm  
    \marginpar{ 
        \fontsize{9}{8}\selectfont
        \fboxsep=1.5mm
        \vspace{0.2cm} \hspace{-0.08cm} \fbox{#1} \vspace{0.2cm} \\
        \fontsize{8}{7.5}
        \selectfont #2
    }
    \marginparwidth=2.5cm
    \normalsize
    \interlineado{1.0}
}

%% Bloque llamativo %%%%%%%%%%%%%%%%%%%%%%%%%%%%%%%%%%%%%%%%%%%%%%%%%%
%% Uso: \importante{tipo}{Texto} % tipo = warning | info | question

\def\importante#1#2{
    \vspace{3mm}
    \small    
    \fboxrule=0.5pt
    \fboxsep=5mm
    \par\noindent\fcolorbox{grisoscuro}{grisclaro}{\parbox{1.3cm}{\resizebox{1cm}{!}{\includegraphics{./img/iconos/#1.png}}}\parbox{0.81\hsize}{#2}}
    \vspace{3mm}
    \fboxsep=1.5mm
    \normalsize
}

%% Insercion de imagen como figura %%%%%%%%%%%%%%%%%%%%%%%%%%%%%%%%%%%
%% Uso: \imagenhere{nombreFichero}{Ancho}{Descripcion}{Identificador}

\def\imagenhere#1#2#3#4{
  \begin{figure}[h]
    \begin{center}
%     \resizebox{#2\textwidth}{!}{\includegraphics{#1}}
      \resizebox{#2}{!}{\includegraphics{#1}}
      \ifthenelse{\equal{#3}{}}{}{\caption {#3}}
      \label{#4}
    \end{center}
  \end{figure}
}

%% Letra capital %%%%%%%%%%%%%%%%%%%%%%%%%%%%%%%%%%%%%%%%%%%%%%%%%%%%%
%% Uso: \capital{letra}

\def\capital#1{
    \lettrine[lines=3, lhang=0.00, loversize=0.04]{#1}{}  
}

%% Insercion de citas %%%%%%%%%%%%%%%%%%%%%%%%%%%%%%%%%%%%%%%%%%%%%%%%
%% Uso: \cita{frase}{autor}

\def\cita#1#2{
    \vspace{0.2cm}
    \begin{quote}
        #1
        \begin{flushright}
            {\it #2}
        \end{flushright}
    \end{quote}
    \normalsize
}

%% Inserción de código fuente %%%%%%%%%%%%%%%%%%%%%%%%%%%%%%%%%%%%%%%%
%% Uso: \code{lenguaje}{numero_primera_linea}{caption}{archivo_fuente}

\def\code#1#2#3#4{
  \lstinputlisting[language=#1, firstnumber=#2, caption=#3]{#4}
}

%% Cambiar el interlineado %%%%%%%%%%%%%%%%%%%%%%%%%%%%%%%%%%%%%%%%%%%
%% Uso: \interlineado{factor} % factor 1.0 es el normal, 2.0 doble...

\newcommand{\interlineado}[1]{
    \renewcommand{\baselinestretch}{#1}  % -- Cambiamos interlineado
	 \large\normalsize % ---------------------- Para que cambie de verdad
}

%% Imagen de ancho total %%%%%%%%%%%%%%%%%%%%%%%%%%%%%%%%%%%%%%%%%%%%%
%% Uso: \imagenanchototal{path}{caption}{label}

\def\imagenanchototal#1#2#3{
  \begin{figure}[h]
    \ifodd\thepage
      \hspace{-5.2cm}
    \else
      \hspace{-0.1cm}
    \fi 
    \begin{minipage}{17.1cm}
      \centering
      \includegraphics[width=\textwidth]{#1}
      \ifthenelse{\equal{#2}{}}{}{\caption{#2}}
      \ifthenelse{\equal{#3}{}}{}{\label{#3}}
    \end{minipage}
  \end{figure}
}

\def\imagenanchototalpar#1#2#3{
  \begin{figure}[h]
    \hspace{-5.2cm}
    \begin{minipage}{17.1cm}
      \centering
      \includegraphics[width=\textwidth]{#1}
      \ifthenelse{\equal{#2}{}}{}{\caption{#2}}
      \ifthenelse{\equal{#3}{}}{}{\label{#3}}
    \end{minipage}
  \end{figure}
}

\def\imagenanchototalimpar#1#2#3{
  \begin{figure}[h]
    \hspace{-0.1cm}
    \begin{minipage}{17.1cm}
      \centering
      \includegraphics[width=\textwidth]{#1}
      \ifthenelse{\equal{#2}{}}{}{\caption{#2}}
      \ifthenelse{\equal{#3}{}}{}{\label{#3}}
    \end{minipage}
  \end{figure}
}


%%%%



%% Grafos DOT empotrados %%%%%%%%%%%%%%%%%%%%%%%%%%%%%%%%%%%%%%%%%%%%%
%% Uso: FALTA POR HACER UN ENTORNO PARA ESTO. A CONTINUACION PUEDES VER
%% UN EJEMPLO DE USO:

%% \begin{figure}[h]
%% \begin{center}
%%   \begin{dot2tex}[dot,scale=1,mathmode]
%%     digraph G {
%%       rankdir=TB;
%%       d2ttikzedgelabels = false;
%%       node [style="state"];
%%       edge [lblstyle="auto"];
%%       X [label = "\pi"];
%%       A -> B [label = "\displaystyle\frac{x_2}{\sum y_i}"];
%%       A -> D [label = "7"];
%%       A -> X;
%%       X -> D;
%%     }
%%   \end{dot2tex}
%% \caption{Grafo empotrado}
%% \label{grafo01}
%% \end{center}
%% \end{figure}





%% Más cosillas




%% % ------------------ Macro para aniadir un cuadro sombreado ---------------------
%% %       Uso: \begin{sombreado}{Texto...}\end{sombreado} 
%% % ------------------------------------------------------------------------------

%% \newenvironment{sombreado}[1]%
%% {%\vspace{2mm}
%%  \par\noindent
%%  \fboxrule=1pt
%%  \fboxsep=2mm
%%  \addtolength{\hsize}{-2\fboxsep} 
%%  \addtolength{\hsize}{-2\fboxrule}
%%  \fcolorbox{black}{gris}{\parbox{\hsize}{#1}}
%%  %\vspace{2mm}
%%  }{}



%% % ------------------ Macro para insertar una subimagen con marco -------------------
%% %       Uso: \subimagenmarco{nombreFichero}{Ancho}{Etiqueta}{Identificador}
%% % -----------------------------------------------------------------------------
%% \def\subimagenmarco#1#2#3#4{
%%  \subfigure[#3]{
%%    \resizebox{#2\textwidth}{!}{\linethickness{0.3mm} \frame{\includegraphics{#1}}}
%%    \label{'#4'}
%%  }	
%% }


%% \newenvironment{intro}[1]%
%% {\vspace{-15mm}
%%  \small
%%  \par\noindent
%%  \fboxrule=1pt
%%  \fboxsep=4mm
%%  \fcolorbox{white}{white}{\parbox{0.92\hsize}{\textcolor{grisoscuro}{$\triangle$ \textit{#1}}}}
%%  \vspace{5mm}
%%  \fboxsep=1.5mm
%%  \normalsize
%%  }{}

%% % ------------------ Macro para insertar una imagen cedida por alguien --------
%% %       Uso: \imagen{nombreFichero}{Ancho}{Etiqueta}{Identificador}
%% % -----------------------------------------------------------------------------
%% \def\imagenCopyright#1#2#3#4#5{
%%  \begin{figure}[here]
%%  \begin{center}
%%    \resizebox{#2\textwidth}{!}{\includegraphics{#1}}
%%  \fontsize{7}{7}\selectfont
%%  \begin{flushright}
%% 	 \vspace{-1mm}
%% 	 \textsf{#5}
%%  \end{flushright}
%%  \normalsize 
%%  \vspace{-3mm}
%%  \caption {#3}
%%  \label{'#4'}
%%  \end{center}
%%  \end{figure}
%% }


%% % ------------------ Macro para insertar una imagen con marco
%% %       Uso: \imagenmarco{nombreFichero}{Ancho}{Etiqueta}{Identificador}
%% % -----------------------------------------------------------------------------
%% \def\imagenmarco#1#2#3#4{
%%  \begin{figure}[here]
%%  \begin{center}
%%    \resizebox{#2\textwidth}{!}{\linethickness{0.3mm} \frame{\includegraphics{#1}}}
%%  \ifthenelse{\equal{#3}{}}{}{\caption {#3}}
%%  \label{'#4'}
%%  \end{center}
%%  \end{figure}
%% }

%% \def\imagenanchototal#1#2#3#4#5{
%%   \captionsetup{margin=4pt,font=small,labelfont=bf}
%%  \begin{figure}
%%  \ifthenelse{\equal{#5}{par}}{\hspace{0cm}}{\hspace{-5.3cm}} 
%%  \resizebox{#2}{!}{\includegraphics{#1}}
%%  \ifthenelse{\equal{#3}{}}{}{\caption {#3}}
%%  \label{#4}
%%  \end{figure}
%% }

%% %% ------------------ Macro para insertar una imagen ---------------------------
%% %       Uso: \imagen{nombreFichero}{Ancho}{Etiqueta}{Identificador}
%% % -----------------------------------------------------------------------------
%% \def\imagen#1#2#3#4{
%%   \captionsetup{margin=4pt,font=small,labelfont=bf}
%%  \begin{figure}
%%  \begin{center}
%%    \resizebox{#2\textwidth}{!}{\includegraphics{#1}}
%%  \ifthenelse{\equal{#3}{}}{}{\caption {#3}}
%%  \label{#4}
%%  \end{center}
%%  \end{figure}
%% }

%% % ------------------ Macro para aniadir codigo fuente --------------------
%% %       Uso: \begin{VerbatimEnvironment}[label=\normalsize\textrm{\textbf{titulo}}]
%% % -----------------------------------------------------------------------

%% \DefineVerbatimEnvironment
%%  {MiVerbatim}{Verbatim}
%%  {frame=single, framerule=.3mm, framesep=4mm, samepage=false,
%%    rulecolor=\color{gris}, baselinestretch=.7, fontsize=\small, 
%%    numbers=left, numbersep=2mm, xleftmargin=2mm, xrightmargin=2mm}
 
%% %%%%%%%%%%%%%%%%%%%%% IMAGEN EN EL MARGEN %%%%%%%%%%%%%%%%%%%%%%%%%%%%%%%

%% \def\slide#1{
%%   \ifodd\thepage \marginparsep=2.5cm \else \marginparsep=1cm \fi 
%%   \marginpar{ 
%%     \resizebox{1.6\marginparwidth}{!}{\linethickness{0.3mm} \frame{\includegraphics{#1}}}
%%   }
%% }
